\documentclass[
  ngerman,
  a4paper,
  12pt
]{article}

\usepackage[utf8]{inputenc}
\usepackage[english, german]{babel}
\usepackage[left=2.5cm, right=2.5cm]{geometry}
\usepackage{csquotes}
\usepackage{setspace}
\usepackage[dvipsnames]{xcolor}

\usepackage[shortcuts=ac]{glossaries-extra}

\usepackage[unicode]{hyperref}
\hypersetup{
  linkcolor = Red,
  filecolor = Magenta,
  urlcolor = Green,
  citecolor = Cyan
}

\usepackage[style = authoryear-ibid]{biblatex}
\addbibresource{preamble/main.bib}

\usepackage{enumitem}
\setlist{
  topsep=0pt,
  itemsep=0pt,
  parsep=0pt,
  align=left,
  leftmargin=*
}

\usepackage{tikz}
\usetikzlibrary{
  positioning,
  calc,
  arrows.meta,
  fit,
  matrix
}

\usepackage{siunitx}
\usepackage{amsmath}
\usepackage{amsthm}
\usepackage{amssymb}
\usepackage{mathtools}
\usepackage{xparse}

\newtheorem{lemma}{Lemma}
\newcommand*{\lemmaautorefname}{Lemma}
\newtheorem{corollary}{Kontrollar}
\newcommand*{\corollaryautorefname}{Kontrollar}
\newtheorem{folgerung}{Folgerung}
\newcommand*{\folgerungautorefname}{Folgerung}

\counterwithin*{equation}{section}
\counterwithin*{equation}{subsection}

% Sets
\newcommand*{\swhole}{\mathbb{N}}
\newcommand*{\snatural}{\mathbb{N}^\times}
\newcommand*{\sintegers}{\mathbb{Z}}
\newcommand*{\srational}{\mathbb{Q}}
\newcommand*{\sprime}{\mathbb{P}}

% Math helpers

% - Functions
\newcommand*{\primecountfunc}[1]{w_p(#1)}
\newcommand*{\primesequencecountfunc}[1]{\Pi{}(#1)}
\NewDocumentCommand{\phifunc}{m e{_}}{%
  \IfNoValueTF{#2}{\varphi(#1)}{\varphi_{#2}(#1)}%
}

% - Others
\DeclarePairedDelimiter{\floor}{\lfloor}{\rfloor}
\newcommand*{\defined}{:=}
\newcommand*{\coprime}{\perp}
\newcommand*{\implication}[2]{\asroman{#1}) $\Rightarrow$ \asroman{#2})}
\newcommand*{\iimplication}[2]{\asroman{#1}) $\Leftrightarrow$ \asroman{#2})}
\newcommand*{\ggt}[2]{\text{ggT}(#1,#2)}
\newcommand*{\kgv}[2]{\text{kgV}(#1,#2)}
\newcommand*{\idealintergers}[1]{\sintegers{}#1}
\newcommand*{\associated}[2]{#1 \sim #2}

% Environment
\makeatletter
\newcommand*{\noindentafterenv}[1]{%
  \AfterEndEnvironment{#1}{\par\@afterindentfalse\@afterheading}}
\makeatother

\newenvironment*{widemath}% Environment name
{\addtolength{\jot}{0.25\baselineskip}}% Begin command
{}% End command
\noindentafterenv{widemath}

% General
\makeatletter
\newcommand*{\asroman}[1]{\romannumeral #1}
\newcommand*{\Asroman}[1]{\expandafter\@slowromancap\romannumeral #1@}
\makeatother
\newcommand*{\ad}[1]{ad #1):}
\newcommand*{\bignewline}{\\[0.35\baselineskip]}

% Glossary
\newacronym{dh}{d.\,h.}{das heißt}
\newacronym{obda}{o.\,B.\,d.\,A.}{ohne Beschränkung der Allgemeinheit}


\begin{document}
\pagenumbering{Roman}
\onehalfspacing
\tableofcontents
\newpage
\pagenumbering{arabic}

\part{Elementare Zahlentheorie}
Aufgaben aus dem Buch: \fullcite{book:zahlentheorie}.

\section{Seite 28}

\subsection{Aufgabe 1}
Seien $a,b,c$ Ziffern aus der Menge $\{0,1,2,\dotsc,9\}$ und $a \neq 0$.
Zeigen Sie: 13 teilt die natürliche Zahl $abcabc$ (Zifferndarstellung).
\begin{proof}
  Es werden die Differenzen betrachtet, wenn sich $a, b, c$ um einen Wert verändern:
  \begin{alignat*}{2}
    a & = 1 \enspace\text{nach}\enspace a &  & = 2 : \triangle 100100            \\
    b & = 0 \enspace\text{nach}\enspace b &  & = 1 : \triangle \hphantom{0}10010 \\
    c & = 0 \enspace\text{nach}\enspace c &  & = 1 : \triangle \hphantom{00}1001
  \end{alignat*}
  Es ist zu sehen $13 \mid 1001$ mit $1001 = 13 \cdot 77$.
  Hieraus folgt $13 \mid 10010, 13 \mid 100100$ und damit auch
  $13 \mid 1001 \cdot v_1 + 10010 \cdot v_2 + 100100 \cdot v_3 = abcabc$
  mit $v_{1,2,3} \in \{0,1,2,\dotsc,9\}$.
\end{proof}

\newpage
\subsection{Aufgabe 2}
Sei $n$ eine natürliche Zahl, $n > 1$. Beweisen Sie: Aus $n \mid (n - 1)! + 1$
folgt $n \in \PrimeNumbers{}$.
\begin{proof}
  Ist $n$ eine zusammengesetzte Zahl $n = ab$ mit $a,b > 1$, dann gilt
  $a \mid (n - 1)!$ und dadurch $a \nmid (n - 1)! + 1$ weshalb ebenfalls
  $n \nmid (n - 1)! + 1$. \\[4pt]
  \emph{Der restliche Beweis mit dem Satz von Wilson}
\end{proof}

\newpage
\subsection{Aufgabe 3}
Sei $p_n$ die $n$-te Primzahl, d.\,h. $p_1 = 2$, $p_2 = 3$ usw. Zeigen Sie:
$p_n \leq 2^{2^{n - 1}}$ für alle $n \geq 1$.
\begin{proof}
\end{proof}

\newpage
\subsection{Aufgabe 4}
Sei $p$ eine Primzahl. Beweisen Sie: $p$ ist ein Teiler von $\binom{p}{v}$
für $1 \leq v < p$.
\begin{proof}
  Per Definition gilt:
  \begin{equation*}
    \binom{p}{v} = \frac{p(p-1) \cdot \ldots \cdot (p - v + 1)}{v!}
  \end{equation*}
  Es gilt außerdem:
  \begin{equation*}
    \binom{n}{v} \in \WholeNumbers{} \quad
    \text{für alle} \enspace n,v \in \WholeNumbers{}
  \end{equation*}
  Die Primzerlegung des Nenners muss
  vollständig in der des Zählers vorhanden sein.
  Wegen $p > v$ ist $p$ jedoch niemals Teil dieser Zerlegung und
  kann im Zähler nicht gekürzt werden.
  Es folgt $p \mid \binom{p}{v}$.
\end{proof}
\noindent
Es kann nun eine verallgemeinerte Eigenschaft der eben beschriebenen
Teilbarkeit beschrieben werden. Der Beweis des folgenden Lemmas wir
in der nächsten Aufgabe hilfreich sein.
\begin{lemma}
  Sei $p$ eine Primzahl. Dann gilt
  \begin{equation*}
    p \mid \binom{p^n}{v} \quad \text{für alle} \enspace n \in \WholeNumbers{}
    \enspace \text{und} \enspace 1 \leq v < p^n
  \end{equation*}
\end{lemma}
\begin{proof}
  Die folgende Identität ist korrekt:
  \begin{widemath}
    \begin{equation*}
      \begin{aligned}
        \binom{p^n}{v}  & = \frac{p^n}{v}\binom{p^n - 1}{v - 1} \\
        v\binom{p^n}{v} & = p^n\binom{p^n - 1}{v - 1}
      \end{aligned}
    \end{equation*}
  \end{widemath}
  Es ist somit zu sehen, dass $p^n \mid v\binom{p^n}{v}$.
  \begin{enumerate}
    \item Sind $p$ und $v$ teilerfremd, gilt $p^n \mid \binom{p^n}{v}$ und es
          bleibt nichts mehr zu zeigen
    \item Anderenfalls ist $v = p^{n-a}q$ mit $a \in \WholeNumbers{}$ und $0 < a \leq n$
          (bemerke $p$ und $q$ sind teilerfremd und $a > 0$ wegen $v < p^n$)
  \end{enumerate}
  \noindent
  Es gilt daher
  \begin{widemath}
    \begin{align*}
      p^{n-a}q\binom{p^n}{v} & = p^n\binom{p^n - 1}{v - 1} \\
      q\binom{p^n}{v}        & = p^a\binom{p^n - 1}{v - 1}
    \end{align*}
  \end{widemath}
  \noindent
  und somit $p^a \mid \binom{p^n}{v}$.
  Außerdem gilt $p \mid p^a$ und letztendlich $p \mid \binom{p^n}{v}$.
\end{proof}

\newpage
\subsection{Aufgabe 5}
Seien $p \in \PrimeNumbers{}$, $n \in \NaturalNumbers{}$ und $a,b \in \Integers{}$. Zeigen Sie durch Induktion
nach $n$: $p$ ist ein Teiler von
$((a + b)^{p^n} - (a^{p^n} + b^{p^n}))$.
\begin{proof}
  Es ist $B$ die Menge aller Zahlen $n \in \NaturalNumbers{}$,
  sodass für alle $a,b \in \Integers{}$ die behauptete Teilbarkeit richtig ist.
  Es ist $1 \in B$, denn es gilt nach dem Binomischen Lehrsatz
  \parencite[19]{book:zahlentheorie}:
  \begin{widemath}
    \begin{equation*}
      \begin{aligned}
        (a + b)^p - (a^p + b^p) & =
        \left[a^p + \binom{p}{1}a^{p-1}b + \dotsb + \binom{p}{p - 1}ab^{p-1} + b^p\right] - (a^p + b^p) \\
                                & = \binom{p}{1}a^{p-1}b + \dotsb + \binom{p}{p - 1}ab^{p-1}
      \end{aligned}
    \end{equation*}
  \end{widemath}
  Jeder Summand ist als Produkt von $\binom{p}{1},\dotsc,\binom{p}{p - 1}$ durch $p$ teilbar.
  Sei $n \in B$. Um $n + 1 \in B$ zu verifizieren, rechnen wir wie folgt:
  \begin{equation*}
    \begin{aligned}
      (a + b)^{p^{n+1}} - (a^{p^{n+1}} + b^{p^{n+1}}) & =
      \binom{p^{n+1}}{1}a^{p^{n+1}-1}b + \dotsb + \binom{p^{n+1}}{p^{n+1} - 1}ab^{p^{n+1}-1}
    \end{aligned}
  \end{equation*}
  Wieder ist zu sehen, dass jeder Term als ein Vielfaches von
  $\binom{p^{n+1}}{1},\dotsc,\binom{p^{n+1}}{p^{n+1} - 1}$ durch $p$ teilbar ist.
\end{proof}

\newpage
\subsection{Aufgabe 6}
Sei $n \geq 2$ eine natürliche Zahl. Zeigen Sie: $n^4 + 4^n$ ist keine Primzahl.
\begin{proof}
\end{proof}

\newpage
\section{Seite 33}

\subsection{Aufgabe 4}
Es seien $a, b$ natürliche Zahlen, für die gilt:
$a \mid b^2, b^2 \mid a^3, a^3 \mid b^4, b^4 \mid a^5, \dots$.\\
Zeigen sie: $a = b$.

\begin{proof}
Es gibt $v_i \in \MN{}$ mit
$b^2 = av_1, a^3 = b^2v_2, b^4 = a^3v_3, a^5 = b^4v_4, \dotsc$.
Es sind
\begin{align*}
  a & = p_1^{m_1} \cdot p_2^{m_2} \cdot \ldots \cdot p_r^{m_r} \\
  b & = p_1^{\mu_1} \cdot p_2^{\mu_2} \cdot \ldots \cdot p_r^{\mu_r}
\end{align*}
die kanonischen Primzerlegungen von $a$ und $b$.
Für $a = b$ ist die Behauptung offensichtlich richtig.
Angenommen $a \neq b$ und es werden zwei Fälle unterschieden: \\[4pt]
$0 < a < b$: Die Zahl $a$ besitzt als Teiler von $b^2$ einen Primfaktor
$p^{\mu - n} := p_i^{\mu_i - n}$ mit $0 < n \leq \mu$ und $i = 1,\dotsc,r$.
Es gilt $p^{\mu - n} \mid b^2$
d.\,h. $b^2 = p^{\mu - n}v_1$, was äquivalent zu
$p^{2\mu} = p^{\mu - n} \cdot p^{\mu + n}$ ist.
Wird dieses Schema fortgeführt, entstehen die Gleichungen
\begin{align*}
p^{3\mu - 3n} & = p^{2\mu}      \cdot p^{\mu - 3n} \\
p^{4\mu}      & = p^{3\mu - 3n} \cdot p^{\mu + 3n} \\
p^{5\mu - 5n} & = p^{4\mu}      \cdot p^{\mu - 5n} \\
p^{6\mu}      & = p^{5\mu - 5n} \cdot p^{\mu + 5n} \\
\vdots \\
p^{2k\mu} & = p^{(2k - 1)\mu - (2k - 1)n} \cdot p^{\mu + (2k - 1)n}  \\
\label{eq:fall1} \tag{$*$}
p^{(2k + 1)\mu - (2k + 1)n} & = p^{2k\mu} \cdot p^{\mu - (2k + 1)n}
\end{align*}
mit $k \in \MNx{}$. Im allgemeinen lässt sich aus \eqref{eq:fall1}
die Ungleichung
\begin{equation*}
\mu - 2kn - n \geq 0
\end{equation*}
ableiten. $k$ ist frei wählbar, man setze $k = \mu$ und führt
die ursprüngliche Behauptung mit $(1 - 2n)\mu - n \geq 0$ zum Widerspruch. \\[4pt]
$a > b$: Die Zahl $a$ besitzt als Teiler von $b^2$ einen Primfaktor
$p^{\mu + n} := p_i^{\mu_i + n}$ mit $0 < n \leq \mu$ und $i = 1,\dotsc,r$.
Wäre $n > \mu$ gilt $a \nmid b^2$ und es bleibt nichts mehr zu zeigen.
Wie zuvor gilt $p^{\mu + n} \mid b^2$,
woraus nach demselben Prinzip die Gleichung
$p^{2\mu} = p^{\mu + n} \cdot p^{\mu - n}$
und letztendlich
\begin{align*}
\label{eq:fall2} \tag{$**$}
p^{2k\mu} & = p^{(2k - 1)\mu + (2k - 1)n} \cdot p^{\mu - (2k - 1)n} \\
p^{(2k + 1)\mu + (2k + 1)n} & = p^{2k\mu} \cdot p^{\mu + (2k + 1)n}
\end{align*}
folgt. Aus \eqref{eq:fall2} kann die Ungleichung
\begin{equation*}
\mu - 2kn + n \geq 0
\end{equation*}
abgeleitet werden. Man setze $k = 2\mu$ und erzeugt mit
$(1 - 4n)\mu + n \geq 0$ auch in diesem Fall einen Widerspruch.
\end{proof}
\newpage
\section{Seite 53}

\subsection{Aufgabe 1}
Sei $p$ eine Primzahl, $a,b$ seien von Null verschiedene rationale Zahlen,
$a + b \neq 0$. Zeigen Sie: $\vielfachfunktion{a + b} \geq \min{(\vielfachfunktion{a}, \vielfachfunktion{b})}$
\begin{proof}
  Sei $m = \min{(\vielfachfunktion{a}, \vielfachfunktion{b})}$. Es gilt $p^m \mid a$, $p^m \mid b$
  und damit auch $p^m \mid a + b$. Wir schreiben $a + b = p^m \cdot v$
  und zeigen durch umformen:
  \begin{align*}
    \vielfachfunktion{a + b} & = \vielfachfunktion{p^m \cdot v}                                            \\
                             & = \vielfachfunktion{p^m} + \vielfachfunktion{v}                             \\
                             & = m + \vielfachfunktion{v}                                                  \\
                             & = \min{(\vielfachfunktion{a}, \vielfachfunktion{b})} + \vielfachfunktion{v}
  \end{align*}
  Es ist zu sehen $\vielfachfunktion{a + b} \geq \min{(\vielfachfunktion{a}, \vielfachfunktion{b})}$.
\end{proof}

\subsection{Aufgabe 2}
Für $x$ reell bezeichne $\floor{x}$ die größte ganze Zahl $m$ mit $m \leq x$.
Zeigen Sie, dass für $p$ eine Primzahl und $n \in \WholeNumbers{}$ beliebig gilt:
\begin{equation*}
  \vielfachfunktion{n!} = \sum_{i=0}^{\infty} \floor*{\frac{n}{p^i}}
\end{equation*}
\begin{proof}
\end{proof}

\subsection{Aufgabe 3}
Seien $n \in \NaturalNumbers{}$, $a_1,\dotsc,a_n \in \Integers{}$.
Die reelle Zahl $x$ erfülle $x^n + a_1x^{n-1} + \dotsb + a_{n - 1}x + a_n = 0$.
Zeigen Sie: $x$ ist entweder irrational oder ganz.
\begin{proof}
\end{proof}

\subsection{Aufgabe 4}
Seien $q_1,\dotsc,q_s$ Primzahlen, $b \defined{} q_1 \cdot q_2 \dotsm q_s \in \WholeNumbers{}$
sowie $m_1,\dotsc,m_k \in \NaturalNumbers{}$ derart, dass gilt:
$\frac{1}{b} = \frac{1}{m_1} + \frac{1}{m_2} + \dotsb + \frac{1}{m_k}$.
Zeigen Sie: Jede Zahl $q_i, 1 \leq i \leq s$, teilt wenigstens eine der Zahlen $m_1,\dotsc,m_k$.
\begin{proof}
\end{proof}

\subsection{Aufgabe 5}
Berechnen Sie die Fibonaccidarstellung des Bruches $\frac{21}{23}$.
\begin{proof}
\end{proof}

\subsection{Aufgabe 6}
Zeigen Sie: Es gibt keine ägyptische Bruchdarstellung
$\frac{21}{23} = \frac{1}{n_1} + \frac{1}{n_2} + \dotsb + \frac{1}{n_k}$,
$1 < n_1 < n_2 < \dotsb < n_k$, mit höchstens 3 Stammbrüchen
(d.\,h. notwendig $k \geq 4$).
\begin{proof}
\end{proof}

\subsection{Aufgabe 7}
Beweisen Sie die angegebene Eindeutigkeitsaussage für die Fibonaccidarstellung
\parencite[53]{book:zahlentheorie}.
\begin{proof}
\end{proof}

\newpage
\section{(70) Größter gemeinsamer Teiler}

\subsection{Aufgabe 1}
Seinen $a,m,n \in \snatural$. Bestimmen Sie den größten gemeinsamen Teiler
von $a^m - 1$ und $a^n - 1$.
\begin{proof}
\end{proof}

\subsection{Aufgabe 2}
Seien $a,b \in \snatural$ teilerfremd und $c \in \swhole$ so,
dass gilt: $a \mid c$ und $b \mid c$. Zeigen Sie: $ab \mid c$.
\begin{proof}
  Das Kriterium für paarweise Teilerfremdheit \parencite[50]{book:zahlentheorie}
  enthält als einfache
  \begin{folgerung}
    \label{folgerung:70:1}
    Seien $a,b \in \sintegers$ zwei teilerfremde Zahlen, dann ist
    $\min{(\vielfachfunktion{a}, \vielfachfunktion{b})} = 0$
    für alle $p \in \sprime$.
  \end{folgerung}
  \begin{proof}
    Wäre die Vielfachheitsfunktions mit $p$ für beide $a,b > 0$.
    Dann ist genau dieses $p$ ein gemeinsamer Teiler von $a$ und $b$.
  \end{proof}
  \noindent
  Es gilt $\vielfachfunktion{a} \leq \vielfachfunktion{c}$,
  $\vielfachfunktion{b} \leq \vielfachfunktion{c}$ für alle $p \in \sprime$
  nach dem Teilbarkeitskriterium \parencite[50]{book:zahlentheorie}.
  Es ist $ab = \sum_{p} p^{\vielfachfunktion{a} + \vielfachfunktion{b}}$
  mit $p \in \sprime$ die Primzerlegung von $ab$.
  Da $a$ und $b$ teilerfremd sind, gilt nach
  \autoref{folgerung:70:1} $\vielfachfunktion{a} + \vielfachfunktion{b}
    \leq \vielfachfunktion{c}$. Es folgt $ab \mid c$.
\end{proof}

\subsection{Aufgabe 3}
Seien $a,b \in \snatural$. Zeigen Sie: $\ggt{a + b}{a - b} \geq \ggt{a}{b}$.
\begin{proof}
\end{proof}

\subsection{Aufgabe 4}
\label{70:4}
Seien $a,b,m \in \sintegers$. Zeigen Sie die Äquivalenz folgender Aussagen:
\begin{enumerate}[label=\roman*)]
  \item Es gibt eine ganze Zahl $x$ mit $m \mid (ax - b)$
  \item $\ggt{a}{m} \mid b$
\end{enumerate}
\begin{proof}
  \implication{1}{2}: Sei $t = \ggt{a}{m}$. Es gilt $t \mid a$, $t \mid m$ und
  aus letzterem $t \mid ax - b$. Weiter gilt nach den Rechenregeln zur Teilbarkeit
  $t \mid b$ und dies erledigt die Beweisrichtung.\\
  \implication{2}{1}: $\ggt{a}{m}$ liefert die Gleichung $t = ra + sm$
  mit $r,s \in \sintegers$. Aus $\ggt{a}{m} \mid b$ folgt mit $v \in \sintegers$:
  \begin{equation*}
    \begin{aligned}
      b      & = tv = rva + svm               \\
      svm    & = b - rva \qquad x \defined rv \\
      (-sv)m & = ax - b
    \end{aligned}
  \end{equation*}
  Daher $m \mid (ax - b)$.
\end{proof}

\subsection{Aufgabe 5}
Seien $m,n \in \sintegers$ teilerfremd, $k \defined mn$ sowie
$a,b \in \sintegers$ beliebig. Zeigen Sie (unter Verwendung von \autoref{70:4}):
\begin{enumerate}[label=\alph*)]
  \item Es gibt eine ganze Zahl $u$ mit $m \mid (u - a)$ und $n \mid (u - b)$
  \item Für eine ganze Zahl $x$ sind äquivalent:
        \begin{enumerate}[label=\roman*)]
          \item $m \mid (x - a)$ und $n \mid (x - b)$
          \item $k \mid (x - u)$
        \end{enumerate}
\end{enumerate}
\begin{proof}
  \begin{enumerate}[label=\alph*)]
    \item Es ist $\textcolor{Red}{p}m = u - a$ und
          $\textcolor{Green}{q}n = u - b$ mit $p,q \in \sintegers$.
          D.\,h. $u$ ist die Lösung der Gleichung
          $\textcolor{Red}{p}m - \textcolor{Green}{q}n = b - a$.
          Nach Voraussetzung existiert $rm + sn = 1$ mit $r,s \in \sintegers$.
          Bemerke die Terme $rm$ und $sn$ haben zwangsweise
          unterschiedliche Vorzeichen. Nach Multiplikation mit
          $b - a$ entsteht daher
          \begin{equation}
            \label{70:1}
            \textcolor{Red}{(br - ar)}m + \textcolor{Green}{(bs - as)}n = b - a.
          \end{equation}
          Also, es existiert ein $u$ mit
          \begin{equation*}
            u = \textcolor{Red}{(br - ar)}m + a = \textcolor{Green}{(bs - as)}n + b.
          \end{equation*}
    \item \autoref{70:1} gibt eine Lösung für $u$.
          Es ist leicht hierdurch alle anderen Lösungen anzugeben.
          Mit dem Wissen des Vorzeichenverhaltens von \eqref{70:1},
          rechne man mit $v \in \sintegers$ wie folgt:
          \begin{equation*}
            \textcolor{Red}{(br - ar + vn)}m + \textcolor{Green}{(bs - as + vm)}n
            = b - a + (vmn - vmn)
          \end{equation*}
          Sind also $x = \textcolor{Red}{(br - ar + v_1n)}m + a$ und
          $u = \textcolor{Red}{(br - ar + v_2n)}m + a$ mit $v_1 \neq v_2 \in \sintegers$
          zwei Lösungen der Teilbarkeit,
          dann ist $x - u = v_1mn - v_2mn = (v_1 - v_2)mn$.
          Mit $k \defined mn$ gilt also $k \mid x - u$.
  \end{enumerate}
\end{proof}

\subsection{Aufgabe 6}
\begin{enumerate}[label=\alph*)]
  \item Seien $\mathfrak{a},\mathfrak{b}$ zwei Ideale in $\sintegers$. Zeigen Sie:
        $\mathfrak{a} \cap \mathfrak{b}$ ist wieder ein Ideal in $\sintegers$.
  \item Zeigen Sie: Für ganze Zahlen $a,b,v$ sind folgende Aussagen äquivalent:
        \begin{enumerate}[label=\roman*)]
          \item $v \geq 0$ und
                $\idealintergers{v} = \idealintergers{a} \cap \idealintergers{b}$
          \item $v = \kgv{a}{b}$
        \end{enumerate}
\end{enumerate}

\begin{lemma}
  \label{lemma:70.6.1}
  Es seien $a,b \in \sintegers$ zwei Zahlen derart, dass für das von ihnen
  erzeugte Hauptideal $\idealintergers{a}, \idealintergers{b}$ gilt:
  $\idealintergers{a} \subseteq \idealintergers{b}$.
  Dann ist notwendigerweise $b \mid a$.
\end{lemma}
\begin{proof}
  Es wird \obda{} angenommen $a,b \geq 0$.
  Der Fall für $b = a$ und $a = 0$ ist klar. Ist $b = 0$, so muss $a$
  als einzige Teilmenge ebenfalls $0$ sein. Es kann niemals gelten $b > a$, denn
  dann wäre $a \notin \idealintergers{b}$. Es muss also sein $b < a$.
  Es ist $a \in \idealintergers{b}$ und
  es gilt somit die Gleichung $bx = a$ mit $x \in \sintegers$.
  Es folgt $b \mid a$.
\end{proof}
\begin{proof}
  \begin{enumerate}[label=\alph*)]
    \item Wir zeigen, dass $\mathfrak{a} \cap \mathfrak{b}$ die Bedingungen
          der Definition eines Ideals in $\sintegers$ erfüllt
          \parencite[60]{book:zahlentheorie}.\\
          \ad{1} Angenommen $a,b \in \mathfrak{a} \cap \mathfrak{b}$.
          Es wird gezeigt, dass auch $a - b \in \mathfrak{a} \cap \mathfrak{b}$.
          Per Annahme wissen wir $a,b \in \mathfrak{a}$ und $a,b \in \mathfrak{b}$.
          Nach Idealdefinition ist somit ebenfalls
          $a - b \in \mathfrak{a}$ und $a - b \in \mathfrak{b}$.
          Es folgt $a - b \in \mathfrak{a} \cap \mathfrak{b}$.\\
          \ad{2} Angenommen $a \in \mathfrak{a} \cap \mathfrak{b}$.
          Es wird gezeigt, dass auch $xa \in \mathfrak{a} \cap \mathfrak{b}$
          mit $x \in \sintegers$. Per Annahme wissen wir
          $a \in \mathfrak{a}$ und $a \in \mathfrak{b}$.
          Nach Idealdefinition ist somit ebenfalls
          $xa \in \mathfrak{a}$ und $xa \in \mathfrak{b}$.
          Es folgt $xa \in \mathfrak{a} \cap \mathfrak{b}$.
    \item Aus $\idealintergers{v} = \idealintergers{a} \cap \idealintergers{b}$
          folgt $\idealintergers{v} \subseteq \idealintergers{a}$,
          $\idealintergers{v} \subseteq \idealintergers{b}$ und nach \autoref{lemma:70.6.1}
          also $a | v$ und $b | v$. Die Zahl $v$ ist somit
          ein gemeinsames Vielfaches von $a$ und $b$.
          Angenommen $c$ ist ein weiteres gemeinsames Vielfaches von $a$ und $b$,
          dann gilt $a | c$ und $b | c$.
          Wieder nach \autoref{lemma:70.6.1} ist also
          $\idealintergers{c} \subseteq \idealintergers{a}$ und
          $\idealintergers{c} \subseteq \idealintergers{b}$.
          Es folgt die logische Schlussfolgerung
          \begin{equation*}
            (\idealintergers{c} \subseteq \idealintergers{a}) \wedge
            (\idealintergers{c} \subseteq \idealintergers{a}) \wedge
            (\idealintergers{a} \cap \idealintergers{b} = \idealintergers{v})
            \Rightarrow \idealintergers{c} \subseteq \idealintergers{v}.
          \end{equation*}
          Es gilt also $v \mid c$, sowie $a \mid v$ und $b \mid v$.
          Die Zahl $v$ erfüllt somit alle Eingenschaften des
          kleinsten gemeinsamen Vielfaches von $a$ und $b$.
  \end{enumerate}
\end{proof}

\subsection{Aufgabe 7}
Seien $a,b,c \in \snatural$. Zeigen Sie: Es gilt $a^2 + b^2 = c^2$ genau dann,
wenn es $s,u,v \in \snatural$ mit $u > v$ gibt, sodass entweder
$a = 2suv$, $b = s(u^2 - v^2)$, $c = s(u^2 + v^2)$ oder
$a = s(u^2 + r^2)$, $b = 2suv$, $c = s(u^2 + v^2)$.
\begin{proof}
\end{proof}



\printbibliography[title = Literaturverzeichnis]
\addcontentsline{toc}{part}{Literaturverzeichnis}
\end{document}
