\documentclass[
  ngerman,
  a4paper,
  12pt
]{article}

\usepackage[utf8]{inputenc}
\usepackage[english, german]{babel}
\usepackage[left=2.5cm, right=2.5cm]{geometry}
\usepackage{csquotes}
\usepackage{setspace}
\usepackage[dvipsnames]{xcolor}

\usepackage[shortcuts=ac]{glossaries-extra}

\usepackage[unicode]{hyperref}
\hypersetup{
  linkcolor = Red,
  filecolor = Magenta,
  urlcolor = Green,
  citecolor = Cyan
}

\usepackage[style = authoryear-ibid]{biblatex}
\addbibresource{preamble/main.bib}

\usepackage{enumitem}
\setlist{
  topsep=0pt,
  itemsep=0pt,
  parsep=0pt,
  align=left,
  leftmargin=*
}

\usepackage{tikz}
\usetikzlibrary{
  positioning,
  calc,
  arrows.meta,
  fit,
  matrix
}

\usepackage{siunitx}
\usepackage{amsmath}
\usepackage{amsthm}
\usepackage{amssymb}
\usepackage{mathtools}
\usepackage{xparse}

\newtheorem{lemma}{Lemma}
\newcommand*{\lemmaautorefname}{Lemma}
\newtheorem{corollary}{Kontrollar}
\newcommand*{\corollaryautorefname}{Kontrollar}
\newtheorem{folgerung}{Folgerung}
\newcommand*{\folgerungautorefname}{Folgerung}

\counterwithin*{equation}{section}
\counterwithin*{equation}{subsection}

% Sets
\newcommand*{\swhole}{\mathbb{N}}
\newcommand*{\snatural}{\mathbb{N}^\times}
\newcommand*{\sintegers}{\mathbb{Z}}
\newcommand*{\srational}{\mathbb{Q}}
\newcommand*{\sprime}{\mathbb{P}}

% Math helpers

% - Functions
\newcommand*{\primecountfunc}[1]{w_p(#1)}
\newcommand*{\primesequencecountfunc}[1]{\Pi{}(#1)}
\NewDocumentCommand{\phifunc}{m e{_}}{%
  \IfNoValueTF{#2}{\varphi(#1)}{\varphi_{#2}(#1)}%
}

% - Others
\DeclarePairedDelimiter{\floor}{\lfloor}{\rfloor}
\newcommand*{\defined}{:=}
\newcommand*{\coprime}{\perp}
\newcommand*{\implication}[2]{\asroman{#1}) $\Rightarrow$ \asroman{#2})}
\newcommand*{\iimplication}[2]{\asroman{#1}) $\Leftrightarrow$ \asroman{#2})}
\newcommand*{\ggt}[2]{\text{ggT}(#1,#2)}
\newcommand*{\kgv}[2]{\text{kgV}(#1,#2)}
\newcommand*{\idealintergers}[1]{\sintegers{}#1}
\newcommand*{\associated}[2]{#1 \sim #2}

% Environment
\makeatletter
\newcommand*{\noindentafterenv}[1]{%
  \AfterEndEnvironment{#1}{\par\@afterindentfalse\@afterheading}}
\makeatother

\newenvironment*{widemath}% Environment name
{\addtolength{\jot}{0.25\baselineskip}}% Begin command
{}% End command
\noindentafterenv{widemath}

% General
\makeatletter
\newcommand*{\asroman}[1]{\romannumeral #1}
\newcommand*{\Asroman}[1]{\expandafter\@slowromancap\romannumeral #1@}
\makeatother
\newcommand*{\ad}[1]{ad #1):}
\newcommand*{\bignewline}{\\[0.35\baselineskip]}

% Glossary
\newacronym{dh}{d.\,h.}{das heißt}
\newacronym{obda}{o.\,B.\,d.\,A.}{ohne Beschränkung der Allgemeinheit}


\begin{document}
\pagenumbering{Roman}
\onehalfspacing
\tableofcontents
\newpage
\pagenumbering{arabic}

\part{Elementare Zahlentheorie}
Aufgaben aus dem Buch: \fullcite{book:zahlentheorie}.

\section{Seite 28}

\subsection{Aufgabe 1}
Seien $a,b,c$ Ziffern aus der Menge $\{0,1,2,\dotsc,9\}$ und $a \neq 0$.
Zeigen Sie: 13 teilt die natürliche Zahl $abcabc$ (Zifferndarstellung).
\begin{proof}
  Es werden die Differenzen betrachtet, wenn sich $a, b, c$ um einen Wert verändern:
  \begin{align*}
    a & = 1 \Rightarrow a = 2: \triangle 100100 \\
    b & = 0 \Rightarrow b = 1: \triangle 10010  \\
    c & = 0 \Rightarrow c = 1: \triangle 1001
  \end{align*}
  Es ist zu sehen $13 \mid 1001$ \footnote{$1001 = 13 \cdot 77$}.
  Hieraus folgt $13 \mid 10010, 13 \mid 100100$ und damit auch $13 \mid abcabc$.
\end{proof}

\newpage
\subsection{Aufgabe 2}
Sei $n$ eine natürliche Zahl, $n > 1$. Beweisen Sie: Aus $n \mid (n - 1)! + 1$
folgt $n \in \SetP$.
\begin{proof}
  Ist $n$ eine zusammengesetzte Zahl $n = ab$ mit $a,b > 1$, dann gilt
  $a \mid (n - 1)!$ und dadurch $a \nmid (n - 1)! + 1$ weshalb ebenfalls
  $n \nmid (n - 1)! + 1$. \\[4pt]
  \emph{Der restliche Beweis mit dem Satz von Wilson}
\end{proof}

\newpage
\subsection{Aufgabe 3}
Sei $p_n$ die $n$-te Primzahl, d.\,h. $p_1 = 2$, $p_2 = 3$ usw. Zeigen Sie:
$p_n \leq 2^{2^{n - 1}}$ für alle $n \geq 1$.
\begin{proof}
\end{proof}

\newpage
\subsection{Aufgabe 4}
Sei $p$ eine Primzahl. Beweisen Sie: $p$ ist ein Teiler von $\binom{p}{v}$
für $1 \leq v \leq p - 1$.
\begin{proof}
  Obwohl die Zahlen $\binom{n}{v}$ als Brüche definiert sind, gilt stets:
  \begin{equation*}
    \binom{n}{v} \in \SetN{} \quad \text{für alle } n,v \in \SetN{}.
  \end{equation*}
  Dies ist auf die Identität
  \begin{equation*}
    \binom{n - 1}{v - 1} + \binom{n - 1}{v} = \binom{n}{v}
  \end{equation*}
  zurückzuführen, mit der sich jeder Binomialkoeffizient rekursiv als
  die Summe natürlicher Zahlen berechnen lässt.
  Per Definition gilt:
  \begin{equation*}
    \binom{p}{v} := \frac{p(p-1) \cdot \ldots \cdot (p - v + 1)}{v!}
  \end{equation*}
  Die Primzerlegung des Nenners muss vollständig in der des Zählers vorhanden
  sein. Wegen $p > v$ ist $p$ jedoch niemals Teil dieser Zerlegung und
  kann im Zähler nicht gekürzt werden.
  Es folgt $p \mid \binom{p}{v}$.
\end{proof}

\newpage
\subsection{Aufgabe 5}
Seien $p \in \SetP{}, n \in \SetNx{}$ und $a,b \in \SetZ{}$. Zeigen Sie durch Induktion
nach $n$: $p$ ist ein Teiler von
$((a + b)^{p^n} - (a^{p^n} + b^{p^n}))$.
\begin{proof}
\end{proof}

\newpage
\subsection{Aufgabe 6}
Sei $n \geq 2$ eine natürliche Zahl. Zeigen Sie: $n^4 + 4^n$ ist keine Primzahl.
\begin{proof}
\end{proof}
\newpage
\section{Seite 33}

\subsection{Aufgabe 3}
Seien $a$ und $b$ positive natürliche Zahlen mit der Eigenschaft, dass
es keine Primzahl gibt, die zugleich $a$ und $b$ teilt. Beweisen Sie:
Gibt es ein $c \in \SetN{}$ mit $ab = c^2$, so
existieren $x,y \in \SetN{}$ mit $a = x^2$ und $b = y^2$.
\begin{proof}
  Es ist $c$ eine beliebige zusammengesetzte Zahl und
  $c^2 = p_1^{2m_1}p_2^{2m_2} \cdot \ldots \cdot p_r^{2m_r}$
  ihre Primzerlegung. Man überlege jetzt,
  wie diese Faktoren zwischen $a$ und $b$ verteilt sein können.
  Damit keine Primzahl in $a$ oder $b$ gemeinsam vorkommt, müssen die
  Primpotenzen $p_i^{2m_i};\,i = 1,\dotsc,r$ vollständig zwischen
  $a$ und $b$ verteilt sein. Somit sind es immer Quadratzahlen.
\end{proof}
\noindent
Zum Beispiel:
\begin{align*}
  20^2 = 2^45^2 \qquad        & 1) \quad ab = (2^4)(5^2) = 4^2 \cdot 5^2         \\[8pt]
  210^2 = 2^23^25^27^2 \qquad & 1) \quad ab = (2^23^25^2)(7^2) = 30^2 \cdot 7^2  \\
                              & 2) \quad ab = (2^23^2)(5^27^2) = 6^2 \cdot 35^2  \\
                              & 3) \quad ab = (2^2)(3^25^27^2) = 2^2 \cdot 105^2
\end{align*}

\newpage
\subsection{Aufgabe 4}
Es seien $a, b$ natürliche Zahlen, für die gilt:
$a \mid b^2, b^2 \mid a^3, a^3 \mid b^4, b^4 \mid a^5, \dots$.\\
Zeigen sie: $a = b$.
\begin{proof}
  Es sind
  \begin{align*}
    a & = X_1^{m_1} \cdot X_2^{m_2} \cdot \ldots \cdot X_r^{m_r}                                             \\
    b & = Y_1^{n_1} \cdot Y_2^{n_2} \cdot \ldots \cdot Y_r^{n_r} \qquad X_i,Y_i \in \SetP{};\,i = 1,\dotsc,r
  \end{align*}
  die Primzerlegungen von $a$ und $b$. Es ist direkt festzuhalten, dass $X_i = Y_i\,\forall\,i$.
  Hätte $a$ zu $b$ zusätzliche Primfaktoren, verletzt dies das Teilbarkeitskriterium in $a \mid b^2$,
  hätte $a$ abzügliche Primfaktoren, verletzt dies $b^2 \mid a^3$.
  Es bleibt zu zeigen, dass auch die Primpotenzen nicht verschieden sind.
  Angenommen $a \neq b$ und es werden zwei Fälle unterschieden:

  \paragraph{1)}
  Es gilt $0 < a < b$ und $a$ hat somit mindestens einen Primfaktoren
  der Form $X_i^{m_i - s_i}$ mit $0 < s_i < m_i$.
  Für diesen Beweis reicht es genau einen dieser Faktoren zu untersuchen und wir schreiben
  $X^{m - s}$ ohne den Index $i$. Es werden die folgenden Fakten aufgeschrieben:
  \begin{align*}
    X^{m - s}                & \mid X^{2m}                    & X^{2m}                    & = X^{m - s} \cdot X^{m + s}                       \\
    X^{2m}                   & \mid X^{3m - 3s}               & X^{3m - 3s}               & = X^{2m} \cdot X^{m - 3s}                         \\
    X^{3m - 3s}              & \mid X^{4m}                    & X^{4m}                    & = X^{3m - 3s} \cdot X^{m + 3s}                    \\
    X^{4m}                   & \mid X^{5m - 5s}               & X^{5m - 5s}               & = X^{4m} \cdot X^{m - 5s}                         \\
    \vdots                   &                                &                           & \vdots                                            \\
    X^{(2k - 1)m - (2k -1)s} & \mid X^{2km}                   & X^{2km}                   & = X^{(2k - 1)m - (2k -1)s} \cdot X^{m + (2k -1)s} \\
    X^{2km}                  & \mid X^{(2k + 1)m - (2k + 1)s} & X^{(2k + 1)m - (2k + 1)s} & = X^{2km} \cdot X^{m - (2k + 1)s}
  \end{align*}
  Es lassen sich die folgenden Ungleichungen ableiten oder direkt ablesen:
  \begin{equation}
    \label{eq:33.4.1}
    \begin{aligned}
      2km           & \geq 2km - m - 2ks + s \\
      0             & \geq -m - 2ks + s      \\
      m + (2k - 1)s & \geq 0
    \end{aligned}
  \end{equation}
  \begin{equation}
    \label{eq:33.4.2}
    \begin{aligned}
      2km + m - 2ks - s & \geq 2km \\
      m - (2k + 1)s     & \geq 0   \\
    \end{aligned}
  \end{equation}
  \noindent
  Es ist zu sehen, dass Ungleichung \ref{eq:33.4.1} für alle $k,m,s$ wahr ist. In \ref{eq:33.4.2}
  wird $k = m$ gewählt und man führt die ursprüngliche Behauptung mit $(1 - 2s)m -s \geq 0$ zum Widerspruch.
  Der Term $1 - 2s$ ist wegen $s > 0$ immer negativ.

  \paragraph{2)}
  Es gilt $a > b$ und $a$ hat somit mindestens einen Primfaktoren
  der Form $X_i^{m_i + s_i}$ mit $s_i > 0$. Es wird nach demselben Prinzip wie zuvor aufgeschrieben
  \begin{align*}
    X^{(2k - 1)m + (2k -1)s} & \mid X^{2km}                   & X^{2km}                   & = X^{(2k - 1)m + (2k -1)s} \cdot X^{m - (2k - 1)s} \\
    X^{2km}                  & \mid X^{(2k + 1)m + (2k + 1)s} & X^{(2k + 1)m + (2k + 1)s} & = X^{2km} \cdot X^{m + (2k + 1)s}
  \end{align*}
  und die folgenden Ungleichungen abgelesen:
  \begin{align}
    \label{eq:33.4.3}
    m - (2k - 1)s & \geq 0 \\
    \label{eq:33.4.4}
    m + (2k + 1)s & \geq 0
  \end{align}
  Es ist zu sehen, dass Ungleichung \ref{eq:33.4.4} für alle $k,m,s$ wahr ist. In \ref{eq:33.4.3}
  wird $k = m + 1$ gewählt und man führt die ursprüngliche Behauptung
  mit $(1 - 2s)m - s \geq 0$ zum Widerspruch. Es folgt $a = b$.
\end{proof}
\newpage
\section{Seite 53}
Vielfachfunktion
\subsection{Aufgabe 1}
Sei $p$ eine Primzahl, $a,b$ seien von Null verschiedene rationale Zahlen,
$a + b \neq 0$. Zeigen Sie: $\vielfachfunktion{a + b} \geq \min{(\vielfachfunktion{a}, \vielfachfunktion{b})}$
\begin{proof}
  Sei $m = \min{(\vielfachfunktion{a}, \vielfachfunktion{b})}$. Es gilt $p^m \mid a$, $p^m \mid b$
  und damit auch $p^m \mid a + b$. Wir schreiben $a + b = p^m \cdot v$
  und zeigen durch umformen:
  \begin{align*}
    \vielfachfunktion{a + b} & = \vielfachfunktion{p^m \cdot v}                                            \\
                             & = \vielfachfunktion{p^m} + \vielfachfunktion{v}                             \\
                             & = m + \vielfachfunktion{v}                                                  \\
                             & = \min{(\vielfachfunktion{a}, \vielfachfunktion{b})} + \vielfachfunktion{v}
  \end{align*}
  Es ist zu sehen $\vielfachfunktion{a + b} \geq \min{(\vielfachfunktion{a}, \vielfachfunktion{b})}$.
\end{proof}

\subsection{Aufgabe 2}
Für $x$ reell bezeichne $\floor{x}$ die größte ganze Zahl $m$ mit $m \leq x$.
Zeigen Sie, dass für $p$ eine Primzahl und $n \in \WholeNumbers{}$ beliebig gilt:
\begin{equation*}
  \vielfachfunktion{n!} = \sum_{i=0}^{\infty} \floor*{\frac{n}{p^i}}
\end{equation*}
\begin{proof}
\end{proof}

\subsection{Aufgabe 3}
Seien $n \in \NaturalNumbers{}$, $a_1,\dotsc,a_n \in \Integers{}$.
Die reelle Zahl $x$ erfülle $x^n + a_1x^{n-1} + \dotsb + a_{n - 1}x + a_n = 0$.
Zeigen Sie: $x$ ist entweder irrational oder ganz.
\begin{proof}
\end{proof}

\subsection{Aufgabe 4}
Seien $q_1,\dotsc,q_s$ Primzahlen, $b \defined{} q_1 \cdot q_2 \dotsm q_s \in \WholeNumbers{}$
sowie $m_1,\dotsc,m_k \in \NaturalNumbers{}$ derart, dass gilt:
$\frac{1}{b} = \frac{1}{m_1} + \frac{1}{m_2} + \dotsb + \frac{1}{m_k}$.
Zeigen Sie: Jede Zahl $q_i, 1 \leq i \leq s$, teilt wenigstens eine der Zahlen $m_1,\dotsc,m_k$.
\begin{proof}
\end{proof}

\subsection{Aufgabe 5}
Berechnen Sie die Fibonaccidarstellung des Bruches $\frac{21}{23}$.
\begin{proof}
\end{proof}

\subsection{Aufgabe 6}
Zeigen Sie: Es gibt keine ägyptische Bruchdarstellung
$\frac{21}{23} = \frac{1}{n_1} + \frac{1}{n_2} + \dotsb + \frac{1}{n_k}$,
$1 < n_1 < n_2 < \dotsb < n_k$, mit höchstens 3 Stammbrüchen
(d.\,h. notwendig $k \geq 4$).
\begin{proof}
\end{proof}

\subsection{Aufgabe 7}
Beweisen Sie die angegebene Eindeutigkeitsaussage für die Fibonaccidarstellung
\parencite[53]{book:zahlentheorie}.
\begin{proof}
\end{proof}



\printbibliography[title = Literaturverzeichnis]
\addcontentsline{toc}{part}{Literaturverzeichnis}
\end{document}
