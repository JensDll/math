\section{Integritätsringe. Teilbarkeitstheorie in Integritäts\-ringen (110)}

\subsection{Aufgabe 1}
Sei $R$ ein Integritätsring, seien $a,b,c,d \in R$. Zeigen Sie:
\begin{enumerate}[label=\alph*)]
  \item Aus $\associated{a}{b}$ und $\associated{c}{d}$ folgt:
        \enspace $\associated{ac}{bd}$.
  \item Aus $\associated{a}{b}$ und $\associated{ac}{bd}$ und
        $a \neq 0$ folgt: \enspace $\associated{c}{d}$.
\end{enumerate}

\begin{proof}
  \begin{enumerate}[label=\alph*)]
    \item Es folgt direkt aus Rechenregel 3)
          der Teilbarkeit \parencite[23]{book:zahlentheorie}:
          Aus $a \mid b$ und $c \mid d$ folgt $ac \mid bd$;
          aus $b \mid a$ und $d \mid c$ folgt $bd \mid ac$.
    \item Wir zeigen $c \mid d$ aus dem Gegebenen:
          \begin{align*}
             & \text{Es gilt} \quad b \mid a
            \underset{\text{mit $c \in R$}}{\Rightarrow} bc \mid \textcolor{Green}{ac} \\
             & \text{Es gilt} \quad \textcolor{Green}{ac} \mid bd
            \underset{\text{transitiv}}{\Rightarrow} bc \mid bd
            \underset{\text{kürzen}}{\Rightarrow} c \mid d
          \end{align*}
          Wir zeigen $d \mid c$ aus dem Gegebenen:
          \begin{align*}
             & \text{Es gilt} \quad a \mid b
            \underset{\text{mit $d \in R$}}{\Rightarrow} ad \mid \textcolor{Green}{bd} \\
             & \text{Es gilt} \quad \textcolor{Green}{bd} \mid ac
            \underset{\text{transitiv}}{\Rightarrow} ad \mid ac
            \underset{\text{kürzen}}{\Rightarrow} d \mid c
          \end{align*}
  \end{enumerate}
\end{proof}
