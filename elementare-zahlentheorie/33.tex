\section{Seite 33}

\subsection{Aufgabe 4}
Es seien $a, b$ natürliche Zahlen, für die gilt:
$a \mid b^2, b^2 \mid a^3, a^3 \mid b^4, b^4 \mid a^5, \dots$.\\
Zeigen sie: $a = b$.

\begin{proof}
Es gibt $v_i \in \MN{}$ mit
$b^2 = av_1, a^3 = b^2v_2, b^4 = a^3v_3, a^5 = b^4v_4, \dotsc$.
Es sind
\begin{align*}
  a & = p_1^{m_1} \cdot p_2^{m_2} \cdot \ldots \cdot p_r^{m_r} \\
  b & = p_1^{\mu_1} \cdot p_2^{\mu_2} \cdot \ldots \cdot p_r^{\mu_r}
\end{align*}
die kanonischen Primzerlegungen von $a$ und $b$.
Für $a = b$ ist die Behauptung offensichtlich richtig.
Angenommen $a \neq b$ und es werden zwei Fälle unterschieden: \\[4pt]
$0 < a < b$: Die Zahl $a$ besitzt als Teiler von $b^2$ einen Primfaktor
$p^{\mu - n} := p_i^{\mu_i - n}$ mit $0 < n \leq \mu$ und $i = 1,\dotsc,r$.
Es gilt $p^{\mu - n} \mid b^2$
d.\,h. $b^2 = p^{\mu - n}v_1$, was äquivalent zu
$p^{2\mu} = p^{\mu - n} \cdot p^{\mu + n}$ ist.
Wird dieses Schema fortgeführt, entstehen die Gleichungen
\begin{align*}
p^{3\mu - 3n} & = p^{2\mu}      \cdot p^{\mu - 3n} \\
p^{4\mu}      & = p^{3\mu - 3n} \cdot p^{\mu + 3n} \\
p^{5\mu - 5n} & = p^{4\mu}      \cdot p^{\mu - 5n} \\
p^{6\mu}      & = p^{5\mu - 5n} \cdot p^{\mu + 5n} \\
\vdots \\
p^{2k\mu} & = p^{(2k - 1)\mu - (2k - 1)n} \cdot p^{\mu + (2k - 1)n}  \\
\label{eq:fall1} \tag{$*$}
p^{(2k + 1)\mu - (2k + 1)n} & = p^{2k\mu} \cdot p^{\mu - (2k + 1)n}
\end{align*}
mit $k \in \MNx{}$. Im allgemeinen lässt sich aus \eqref{eq:fall1}
die Ungleichung
\begin{equation*}
\mu - 2kn - n \geq 0
\end{equation*}
ableiten. $k$ ist frei wählbar, man setze $k = \mu$ und führt
die ursprüngliche Behauptung mit $(1 - 2n)\mu - n \geq 0$ zum Widerspruch. \\[4pt]
$a > b$: Die Zahl $a$ besitzt als Teiler von $b^2$ einen Primfaktor
$p^{\mu + n} := p_i^{\mu_i + n}$ mit $0 < n \leq \mu$ und $i = 1,\dotsc,r$.
Wäre $n > \mu$ gilt $a \nmid b^2$ und es bleibt nichts mehr zu zeigen.
Wie zuvor gilt $p^{\mu + n} \mid b^2$,
woraus nach demselben Prinzip die Gleichung
$p^{2\mu} = p^{\mu + n} \cdot p^{\mu - n}$
und letztendlich
\begin{align*}
\label{eq:fall2} \tag{$**$}
p^{2k\mu} & = p^{(2k - 1)\mu + (2k - 1)n} \cdot p^{\mu - (2k - 1)n} \\
p^{(2k + 1)\mu + (2k + 1)n} & = p^{2k\mu} \cdot p^{\mu + (2k + 1)n}
\end{align*}
folgt. Aus \eqref{eq:fall2} kann die Ungleichung
\begin{equation*}
\mu - 2kn + n \geq 0
\end{equation*}
abgeleitet werden. Man setze $k = 2\mu$ und erzeugt mit
$(1 - 4n)\mu + n \geq 0$ auch in diesem Fall einen Widerspruch.
\end{proof}