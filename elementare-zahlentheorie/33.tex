\section{Seite 33}

\subsection{Aufgabe 1}
Folgern Sie aus der Eindeutigkeit der Primzerlegung das Fundamentallemma 1.4
\parencite[26]{book:zahlentheorie}.
\begin{proof}
  Die Primzahl $p$ teilt das Produkt zweier Zahlen $a$ und $b$
  \begin{equation*}
    p \mid \underbrace{(X_1^{m_1} X_2^{m_2} \dotsm X_r^{m_r})}_a
    \underbrace{(Y_1^{m_1} Y_2^{m_2} \dotsm Y_s^{m_s})}_b
  \end{equation*}
\end{proof}

\subsection{Aufgabe 2}
Führen Sie für die Menge $E \defined{} \{4k + 1 : k \in \WholeNumbers{}\}$ entsprechende
Betrachtungen durch wie für die Menge $D$ aus der Bemerkung in Abschnitt 2.
Zeigen Sie insbesondere, dass in $E$ die Zerlegung in in $E$
unzerlegbare Elemente nicht eindeutig bis auf Reihenfolge ist.
\begin{proof}
\end{proof}

\subsection{Aufgabe 3}
Seien $a$ und $b$ positive natürliche Zahlen mit der Eigenschaft, dass
es keine Primzahl gibt, die zugleich $a$ und $b$ teilt. Beweisen Sie:
Gibt es ein $c \in \WholeNumbers{}$ mit $ab = c^2$, so
existieren $x,y \in \WholeNumbers{}$ mit $a = x^2$ und $b = y^2$.
\begin{proof}
  Es ist $c$ eine beliebige zusammengesetzte Zahl und
  $c^2 = p_1^{2m_1}p_2^{2m_2} \cdot \ldots \cdot p_r^{2m_r}$
  ihre Primzerlegung. Man überlege jetzt,
  wie diese Faktoren zwischen $a$ und $b$ verteilt sein können.
  Damit keine Primzahl in $a$ oder $b$ gemeinsam vorkommt, müssen die
  Primpotenzen $p_i^{2m_i}$ mit $i = 1,\dotsc,r$ vollständig zwischen
  $a$ und $b$ verteilt sein. Somit sind es immer Quadratzahlen.
\end{proof}
\noindent
Zum Beispiel:
\begin{align*}
  20^2 = 2^45^2 \qquad        & 1) \quad ab = (2^4)(5^2) = 4^2 \cdot 5^2         \\[8pt]
  210^2 = 2^23^25^27^2 \qquad & 1) \quad ab = (2^23^25^2)(7^2) = 30^2 \cdot 7^2  \\
                              & 2) \quad ab = (2^23^2)(5^27^2) = 6^2 \cdot 35^2  \\
                              & 3) \quad ab = (2^2)(3^25^27^2) = 2^2 \cdot 105^2
\end{align*}

\subsection{Aufgabe 4}
Es seien $a, b$ natürliche Zahlen, für die gilt:
$a \mid b^2, b^2 \mid a^3, a^3 \mid b^4, b^4 \mid a^5, \dots$.\\
Zeigen sie: $a = b$.
\begin{proof}
  Es sind $a = X_1^{m_1}X_2^{m_2} \dotsm X_r^{m_r}$ und
  $b = Y_1^{n_1}Y_2^{n_2} \dotsm Y_s^{n_s}$ die Primzerlegungen von $a$ und $b$
  mit Primzahlen $X_1,X_2,\dotsc,X_r$, $Y_1,Y_2,\dotsc,Y_s$.
  Es ist direkt festzuhalten, dass $r = s$ und $X_i = Y_i$ für alle
  $i = 1,\dotsc,r$. Hätte $a$ mehr Primfaktoren wie $b$,
  verletzt dies das Teilbarkeitskriterium \parencite[33]{book:zahlentheorie} in $a \mid b^2$;
  hätte $a$ weniger, verletzt dies $b^2 \mid a^3$.
  Es bleibt zu zeigen, dass auch die Primpotenzen nicht verschieden sind.
  Angenommen $a \neq b$ und es werden zwei Fälle unterschieden:

  \paragraph{1)}
  Es gilt $0 < a < b$ und $a$ hat somit mindestens einen Primfaktoren
  der Form $X_i^{m_i - s_i}$ mit $0 < s_i < m_i$.
  Für diesen Beweis reicht es genau einen dieser Faktoren zu untersuchen und wir schreiben
  $X^{m - s}$ ohne den Index $i$. Es werden die folgenden Fakten aufgeschrieben:
  \begin{align*}
    X^{m - s}                                     & \mid X^{2m}                    & X^{2m}      & = X^{m - s} \cdot X^{m + s}    \\
    X^{2m}                                        & \mid X^{3m - 3s}               & X^{3m - 3s} & = X^{2m} \cdot X^{m - 3s}      \\
    X^{3m - 3s}                                   & \mid X^{4m}                    & X^{4m}      & = X^{3m - 3s} \cdot X^{m + 3s} \\
    X^{4m}                                        & \mid X^{5m - 5s}               & X^{5m - 5s} & = X^{4m} \cdot X^{m - 5s}      \\
                                                  & \vdotswithin{\mid}             &             & \vdotswithin{=}                \\
    X^{(2k - 1)m - (2k -1)s}                      & \mid X^{2km}                   &
    X^{\textcolor{Red}{2km}}                      &
    = X^{\textcolor{Green}{(2k - 1)m - (2k -1)s}} \cdot X^{\textcolor{Cyan}{m + (2k -1)s}}                                        \\
    X^{2km}                                       & \mid X^{(2k + 1)m - (2k + 1)s} &
    X^{\textcolor{Orange}{(2k + 1)m - (2k + 1)s}} &
    = X^{\textcolor{LimeGreen}{2km}} \cdot X^{\textcolor{Fuchsia}{m - (2k + 1)s}}
  \end{align*}
  Es lassen sich die folgenden Ungleichungen ableiten oder direkt ablesen:
  \begin{equation}
    \label{eq:33.4.1}
    \begin{aligned}
           &  & \textcolor{Red}{2km}            & \geq \textcolor{Green}{2km - m - 2ks + s} \\
      \iff &  & 0                               & \geq -m - 2ks + s                         \\
      \iff &  & \textcolor{Cyan}{m + (2k - 1)s} & \geq 0
    \end{aligned}
  \end{equation}
  \begin{equation}
    \label{eq:33.4.2}
    \begin{aligned}
           &  & \textcolor{Orange}{2km + m - 2ks - s} & \geq \textcolor{LimeGreen}{2km} \\
      \iff &  & \textcolor{Fuchsia}{m - (2k + 1)s}    & \geq 0                          \\
    \end{aligned}
  \end{equation}
  \noindent
  Es ist zu sehen, dass Ungleichung \ref{eq:33.4.1} für alle $k,m,s$ wahr ist. In \ref{eq:33.4.2}
  wird $k = m$ gewählt und man führt die ursprüngliche Behauptung mit
  $(1 - 2s)m -s \geq 0$ zum Widerspruch. Der Term $1 - 2s$ ist wegen $s > 0$ immer negativ.

  \paragraph{2)}
  Es gilt $a > b$ und $a$ hat somit mindestens einen Primfaktoren
  der Form $X_i^{m_i + s_i}$ mit $s_i > 0$. Es wird nach demselben Prinzip wie zuvor aufgeschrieben
  \begin{align*}
    X^{(2k - 1)m + (2k -1)s}  & \mid X^{2km}                                         &
    X^{2km}                   & = X^{(2k - 1)m + (2k -1)s}
    \cdot X^{\textcolor{Red}{m - (2k - 1)s}}                                           \\
    X^{2km}                   & \mid X^{(2k + 1)m + (2k + 1)s}                       &
    X^{(2k + 1)m + (2k + 1)s} & = X^{2km} \cdot X^{\textcolor{Green}{m + (2k + 1)s}}
  \end{align*}
  und die folgenden Ungleichungen abgelesen:
  \begin{align}
    \label{eq:33.4.3}
    \textcolor{Red}{m - (2k - 1)s}   & \geq 0 \\
    \label{eq:33.4.4}
    \textcolor{Green}{m + (2k + 1)s} & \geq 0
  \end{align}
  Es ist zu sehen, dass Ungleichung \ref{eq:33.4.4} für alle $k,m,s$ wahr ist. In \ref{eq:33.4.3}
  wird $k = m + 1$ gewählt und man führt die ursprüngliche Behauptung
  mit $(1 - 2s)m - s \geq 0$ zum Widerspruch.\BigNewline
  Es folgt $a = b$.
\end{proof}