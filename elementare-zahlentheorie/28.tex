\section{Seite 28}

\subsection{Aufgabe 1}
Seien $a,b,c$ Ziffern aus der Menge $\{0,1,2,\dotsc,9\}$ und $a \neq 0$.
Zeigen Sie: 13 teilt die natürliche Zahl $abcabc$ (Zifferndarstellung).
\begin{proof}
  Es werden die Differenzen betrachtet, wenn sich $a, b, c$ um einen Wert verändern:
  \begin{alignat*}{2}
    a & = 1 \enspace\text{nach}\enspace a &  & = 2 : \triangle 100100            \\
    b & = 0 \enspace\text{nach}\enspace b &  & = 1 : \triangle \hphantom{0}10010 \\
    c & = 0 \enspace\text{nach}\enspace c &  & = 1 : \triangle \hphantom{00}1001
  \end{alignat*}
  Es ist zu sehen $13 \mid 1001$ mit $1001 = 13 \cdot 77$.
  Hieraus folgt $13 \mid 10010, 13 \mid 100100$ und damit auch
  $13 \mid 1001 \cdot v_1 + 10010 \cdot v_2 + 100100 \cdot v_3 = abcabc$
  mit $v_{1,2,3} \in \{0,1,2,\dotsc,9\}$.
\end{proof}

\newpage
\subsection{Aufgabe 2}
Sei $n$ eine natürliche Zahl, $n > 1$. Beweisen Sie: Aus $n \mid (n - 1)! + 1$
folgt $n \in \PrimeNumbers{}$.
\begin{proof}
  Ist $n$ eine zusammengesetzte Zahl $n = ab$ mit $a,b > 1$, dann gilt
  $a \mid (n - 1)!$ und dadurch $a \nmid (n - 1)! + 1$ weshalb ebenfalls
  $n \nmid (n - 1)! + 1$. \\[4pt]
  \emph{Der restliche Beweis mit dem Satz von Wilson}
\end{proof}

\newpage
\subsection{Aufgabe 3}
Sei $p_n$ die $n$-te Primzahl, d.\,h. $p_1 = 2$, $p_2 = 3$ usw. Zeigen Sie:
$p_n \leq 2^{2^{n - 1}}$ für alle $n \geq 1$.
\begin{proof}
\end{proof}

\newpage
\subsection{Aufgabe 4}
Sei $p$ eine Primzahl. Beweisen Sie: $p$ ist ein Teiler von $\binom{p}{v}$
für $1 \leq v < p$.
\begin{proof}
  Per Definition gilt:
  \begin{equation*}
    \binom{p}{v} = \frac{p(p-1) \cdot \ldots \cdot (p - v + 1)}{v!}
  \end{equation*}
  Es gilt außerdem:
  \begin{equation*}
    \binom{n}{v} \in \WholeNumbers{} \quad
    \text{für alle} \enspace n,v \in \WholeNumbers{}
  \end{equation*}
  Die Primzerlegung des Nenners muss
  vollständig in der des Zählers vorhanden sein.
  Wegen $p > v$ ist $p$ jedoch niemals Teil dieser Zerlegung und
  kann im Zähler nicht gekürzt werden.
  Es folgt $p \mid \binom{p}{v}$.
\end{proof}
\noindent
Es kann nun eine verallgemeinerte Eigenschaft der eben beschriebenen
Teilbarkeit beschrieben werden. Der Beweis des folgenden Lemmas wir
in der nächsten Aufgabe hilfreich sein.
\begin{lemma}
  Sei $p$ eine Primzahl. Dann gilt
  \begin{equation*}
    p \mid \binom{p^n}{v} \quad \text{für alle} \enspace n \in \WholeNumbers{}
    \enspace \text{und} \enspace 1 \leq v < p^n
  \end{equation*}
\end{lemma}
\begin{proof}
  Die folgende Identität ist korrekt:
  \begin{widemath}
    \begin{equation*}
      \begin{aligned}
        \binom{p^n}{v}  & = \frac{p^n}{v}\binom{p^n - 1}{v - 1} \\
        v\binom{p^n}{v} & = p^n\binom{p^n - 1}{v - 1}
      \end{aligned}
    \end{equation*}
  \end{widemath}
  Es ist somit zu sehen, dass $p^n \mid v\binom{p^n}{v}$.
  \begin{enumerate}
    \item Sind $p$ und $v$ teilerfremd, gilt $p^n \mid \binom{p^n}{v}$ und es
          bleibt nichts mehr zu zeigen
    \item Anderenfalls ist $v = p^{n-a}q$ mit $a \in \WholeNumbers{}$ und $0 < a \leq n$
          (bemerke $p$ und $q$ sind teilerfremd und $a > 0$ wegen $v < p^n$)
  \end{enumerate}
  \noindent
  Es gilt daher
  \begin{widemath}
    \begin{align*}
      p^{n-a}q\binom{p^n}{v} & = p^n\binom{p^n - 1}{v - 1} \\
      q\binom{p^n}{v}        & = p^a\binom{p^n - 1}{v - 1}
    \end{align*}
  \end{widemath}
  \noindent
  und somit $p^a \mid \binom{p^n}{v}$.
  Außerdem gilt $p \mid p^a$ und letztendlich $p \mid \binom{p^n}{v}$.
\end{proof}

\newpage
\subsection{Aufgabe 5}
Seien $p \in \PrimeNumbers{}$, $n \in \NaturalNumbers{}$ und $a,b \in \Integers{}$. Zeigen Sie durch Induktion
nach $n$: $p$ ist ein Teiler von
$((a + b)^{p^n} - (a^{p^n} + b^{p^n}))$.
\begin{proof}
  Es ist $B$ die Menge aller Zahlen $n \in \NaturalNumbers{}$,
  sodass für alle $a,b \in \Integers{}$ die behauptete Teilbarkeit richtig ist.
  Es ist $1 \in B$, denn es gilt nach dem Binomischen Lehrsatz
  \parencite[19]{book:zahlentheorie}:
  \begin{widemath}
    \begin{equation*}
      \begin{aligned}
        (a + b)^p - (a^p + b^p) & =
        \left[a^p + \binom{p}{1}a^{p-1}b + \dotsb + \binom{p}{p - 1}ab^{p-1} + b^p\right] - (a^p + b^p) \\
                                & = \binom{p}{1}a^{p-1}b + \dotsb + \binom{p}{p - 1}ab^{p-1}
      \end{aligned}
    \end{equation*}
  \end{widemath}
  Jeder Summand ist als Produkt von $\binom{p}{1},\dotsc,\binom{p}{p - 1}$ durch $p$ teilbar.
  Sei $n \in B$. Um $n + 1 \in B$ zu verifizieren, rechnen wir wie folgt:
  \begin{equation*}
    \begin{aligned}
      (a + b)^{p^{n+1}} - (a^{p^{n+1}} + b^{p^{n+1}}) & =
      \binom{p^{n+1}}{1}a^{p^{n+1}-1}b + \dotsb + \binom{p^{n+1}}{p^{n+1} - 1}ab^{p^{n+1}-1}
    \end{aligned}
  \end{equation*}
  Wieder ist zu sehen, dass jeder Term als ein Vielfaches von
  $\binom{p^{n+1}}{1},\dotsc,\binom{p^{n+1}}{p^{n+1} - 1}$ durch $p$ teilbar ist.
\end{proof}

\newpage
\subsection{Aufgabe 6}
Sei $n \geq 2$ eine natürliche Zahl. Zeigen Sie: $n^4 + 4^n$ ist keine Primzahl.
\begin{proof}
\end{proof}
