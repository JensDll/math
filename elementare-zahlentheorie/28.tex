\section{(28) Natürliche, ganze und rationale Zahlen. Teilbarkeit. Primzahlen}

\subsection{Aufgabe 1}
Seien $a,b,c$ Ziffern aus der Menge $\{0,1,2,\dotsc,9\}$ und $a \neq 0$.
Zeigen Sie: 13 teilt die natürliche Zahl $abcabc$ (Zifferndarstellung).
\begin{proof}
  Es werden die Differenzen betrachtet, wenn sich $a, b, c$ um einen Wert verändern:
  \begin{alignat*}{2}
    a & = 1 \enspace\text{nach}\enspace a &  & = 2 : \triangle 100100            \\
    b & = 0 \enspace\text{nach}\enspace b &  & = 1 : \triangle \hphantom{0}10010 \\
    c & = 0 \enspace\text{nach}\enspace c &  & = 1 : \triangle \hphantom{00}1001
  \end{alignat*}
  Es ist zu sehen $13 \mid 1001$ mit $1001 = 13 \cdot 77$.
  Hieraus folgt $13 \mid 10010, 13 \mid 100100$ und damit auch
  $13 \mid 100100 \cdot a + 10010 \cdot b + 1001 \cdot a = abcabc$.
\end{proof}

\subsection{Aufgabe 2}
Sei $n$ eine natürliche Zahl, $n > 1$. Beweisen Sie: Aus $n \mid (n - 1)! + 1$
folgt $n \in \sprime$.
\begin{proof}
  \begin{lemma}
    \label{lemma:n_neq4_composite}
    Sei $n \in \swhole$ eine zusammengesetzte Zahl, $n \neq 4$. Dann gilt:
    \begin{equation*}
      n \mid (n - 1)!
    \end{equation*}
  \end{lemma}
  \begin{proof}
    Es ist $n = ab$ mit $a,b \geq 2$. Wir können $(n - 1)!$ wie folgt aufschreiben:
    \begin{equation*}
      n = \textcolor{Red}{a}\textcolor{Green}{b} \mid
      1 \cdot 2 \dotsm \textcolor{Red}{a}
      \textcolor{Green}{(a + 1)(a + 2) \dotsm (a + b)}
      \dotsm (ab - 1) = (ab - 1)!
    \end{equation*}
    Das Produkt $\textcolor{Green}{b}$ aufeinanderfolgender Terme
    enthält zwangsweise ein Vielfaches von $\textcolor{Green}{b}$.
    Außerdem enthält $(ab - 1)!$ $\textcolor{Red}{a}$ und somit
    $\textcolor{Red}{a}\textcolor{Green}{b} \mid (ab - 1)!$.\bignewline
    Die obige Schreibweise ist korrekt, denn wir haben
    \begin{equation*}
      \begin{aligned}
             &  & a + b & \leq ab - 1                                     \\
        \iff &  & 0     & \leq ab - a - b - 1                             \\
        \iff &  & 2     & \leq \underbrace{(a - 1)}_{\geq 1}
        \underbrace{\textcolor{Cyan}{(b - 1)}}_{\textcolor{Cyan}{\geq 2}} \\
      \end{aligned}
    \end{equation*}
    mit $a,b \geq 2$ und niemals $a = b = 2$, da $n \neq 4$.
    Also mindestens \textcolor{Cyan}{einer} der beiden $\textcolor{Cyan}{\geq 3}$.
  \end{proof}
  \noindent
  \autoref{lemma:n_neq4_composite} zeigt, dass $n \mid (n - 1)!$ für alle $n$ zusammengesetzt.
  Man kann also schließen, dass alle Zahlen mit der Eigenschaft $n \mid (n - 1)! + 1$
  nicht zusammengesetzt und daher Prim sind.
\end{proof}

\subsection{Aufgabe 3}
Sei $p_n$ die $n$-te Primzahl, d.\,h. $p_1 = 2$, $p_2 = 3$ usw. Zeigen Sie:
$p_n \leq 2^{2^{n - 1}}$ für alle $n \geq 1$.
\begin{proof}
\end{proof}

\subsection{Aufgabe 4}
Sei $p$ eine Primzahl. Beweisen Sie: $p$ ist ein Teiler von $\binom{p}{v}$
für $1 \leq v < p$.
\begin{proof}
  Per Definition gilt:
  \begin{equation*}
    \binom{p}{v} = \frac{p(p-1) \cdot \ldots \cdot (p - v + 1)}{v!}
  \end{equation*}
  Es gilt außerdem:
  \begin{equation*}
    \binom{n}{v} \in \swhole \quad
    \text{für alle} \enspace n,v \in \swhole
  \end{equation*}
  Die Primzerlegung des Nenners muss
  vollständig in der des Zählers vorhanden sein.
  Wegen $p > v$ ist $p$ jedoch niemals Teil dieser Zerlegung und
  kann im Zähler nicht gekürzt werden.
  Es folgt $p \mid \binom{p}{v}$.
\end{proof}
\noindent
Die eben beschriebene Teilbarkeit lässt sich ganz wesentlich verallgemeinern.
Der Beweis des folgenden Lemmas wir in der nächsten Aufgabe hilfreich sein.
\begin{lemma}
  \label{lemma:p_divides_binom}
  Sei $p$ eine Primzahl. Dann gilt
  \begin{equation*}
    p \mid \binom{p^n}{v} \quad \text{für alle} \enspace n \in \swhole
    \enspace \text{und} \enspace 1 \leq v < p^n
  \end{equation*}
\end{lemma}
\begin{proof}
  Die folgende Identität ist korrekt:
  \begin{widemath}
    \begin{equation*}
      \begin{aligned}
        \binom{p^n}{v}  & = \frac{p^n}{v}\binom{p^n - 1}{v - 1} \\
        v\binom{p^n}{v} & = p^n\binom{p^n - 1}{v - 1}
      \end{aligned}
    \end{equation*}
  \end{widemath}
  Es ist somit zu sehen, dass $p^n \mid v\binom{p^n}{v}$.
  \begin{enumerate}
    \item Sind $p$ und $v$ teilerfremd, gilt $p^n \mid \binom{p^n}{v}$ und es
          bleibt nichts mehr zu zeigen \parencite[64]{book:zahlentheorie}
    \item Anderenfalls ist $v = p^{n-a}q$ mit $a \in \swhole$ und $0 < a \leq n$
          (bemerke $p$ und $q$ sind teilerfremd und $a > 0$ wegen $v < p^n$)
  \end{enumerate}
  \noindent
  Es gilt daher
  \begin{widemath}
    \begin{align*}
      p^{n-a}q\binom{p^n}{v} & = p^n\binom{p^n - 1}{v - 1} \\
      q\binom{p^n}{v}        & = p^a\binom{p^n - 1}{v - 1}
    \end{align*}
  \end{widemath}
  \noindent
  und somit $p^a \mid \binom{p^n}{v}$ aufgrund der
  Teilerfremdheit von $p$ und $q$.
  Außerdem gilt $p \mid p^a$ und letztendlich $p \mid \binom{p^n}{v}$.
\end{proof}

\subsection{Aufgabe 5}
Seien $p \in \sprime$, $n \in \snatural$ und $a,b \in \sintegers$. Zeigen Sie durch Induktion
nach $n$: $p$ ist ein Teiler von
$((a + b)^{p^n} - (a^{p^n} + b^{p^n}))$.
\begin{proof}
  Es ist $B$ die Menge aller Zahlen $n \in \snatural$,
  sodass für alle $a,b \in \sintegers$ die behauptete Teilbarkeit richtig ist.
  Es ist $1 \in B$, denn es gilt:
  \begin{widemath}
    \begin{equation*}
      \begin{aligned}
        (a + b)^p - (a^p + b^p) & =
        \underbrace{\left[a^p + \binom{p}{1}a^{p-1}b + \dotsb +
        \binom{p}{p - 1}ab^{p-1} + b^p\right]}_{\text{Binomischer Lehrsatz \parencite[19]{book:zahlentheorie}}} - (a^p + b^p) \\
                                & = \binom{p}{1}a^{p-1}b + \dotsb + \binom{p}{p - 1}ab^{p-1}
      \end{aligned}
    \end{equation*}
  \end{widemath}
  $p$ teilt die Summe, da jeder Summand als ein Vielfaches
  von $\binom{p}{1},\dotsc,\binom{p}{p - 1}$ durch $p$ teilbar ist.
  Sei $n \in B$. Um $n + 1 \in B$ zu verifizieren, rechnen wir wie folgt:
  \begin{equation*}
    \begin{aligned}
      (a + b)^{p^{n+1}} - (a^{p^{n+1}} + b^{p^{n+1}}) & =
      \binom{p^{n+1}}{1}a^{p^{n+1}-1}b + \dotsb + \binom{p^{n+1}}{p^{n+1} - 1}ab^{p^{n+1}-1}
    \end{aligned}
  \end{equation*}
  Nach \autoref{lemma:p_divides_binom} gilt die obige Eigenschaft auch in diesem Fall.
\end{proof}

\subsection{Aufgabe 6}
Sei $n \geq 2$ eine natürliche Zahl. Zeigen Sie: $n^4 + 4^n$ ist keine Primzahl.
\begin{proof}
  Wir formen um:
  \begin{equation*}
    \begin{aligned}
      n^4 + 4^n & = (n^2)^2 + (2^n)^2                                                \\
                & = (n^2 + 2^n)^2 - (22^nn^2)                                        \\
                & = (n^2 + 2^n)^2 - (2^{n+1}n^2)                                     \\
                & = (n^2 + 2^n)^2 - (2^{\frac{n+1}{2}}n)^2
      \qquad \text{bemerke} \enspace a^2 - b^2 = (a + b)(a - b)                      \\
                & = (n^2 + 2^n + 2^{\frac{n+1}{2}}n)(n^2 + 2^n - 2^{\frac{n+1}{2}}n)
    \end{aligned}
  \end{equation*}
  Es ist zu erkennen, dass für ungerade $n$ immer ein Faktor entsteht.
  Für $n$ gerade, ist die Zahl offensichtlich keine Primzahl,
  da $2 \mid n^4 + 4^n \geq 32$.
\end{proof}
