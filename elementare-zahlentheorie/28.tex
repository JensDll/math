\section{Seite 28}

\subsection{Aufgabe 1}
Seien $a,b,c$ Ziffern aus der Menge $\{0,1,2,\dotsc,9\}$ und $a \neq 0$.
Zeigen Sie: 13 teilt die natürliche Zahl $abcabc$ (Zifferndarstellung).
\begin{proof}
  Es werden die Differenzen betrachtet, wenn sich $a, b, c$ um einen Wert verändern:
  \begin{alignat*}{2}
    a & = 1 \enspace\text{nach}\enspace a &  & = 2 : \triangle 100100            \\
    b & = 0 \enspace\text{nach}\enspace b &  & = 1 : \triangle \hphantom{0}10010 \\
    c & = 0 \enspace\text{nach}\enspace c &  & = 1 : \triangle \hphantom{00}1001
  \end{alignat*}
  Es ist zu sehen $13 \mid 1001$ mit $1001 = 13 \cdot 77$.
  Hieraus folgt $13 \mid 10010, 13 \mid 100100$ und damit auch
  $13 \mid 1001 \cdot v_1 + 10010 \cdot v_2 + 100100 \cdot v_3 = abcabc$, $v_{1,2,3} \in \{0,1,2,\dotsc,9\}$.
\end{proof}

\newpage
\subsection{Aufgabe 2}
Sei $n$ eine natürliche Zahl, $n > 1$. Beweisen Sie: Aus $n \mid (n - 1)! + 1$
folgt $n \in \SetP$.
\begin{proof}
  Ist $n$ eine zusammengesetzte Zahl $n = ab$ mit $a,b > 1$, dann gilt
  $a \mid (n - 1)!$ und dadurch $a \nmid (n - 1)! + 1$ weshalb ebenfalls
  $n \nmid (n - 1)! + 1$. \\[4pt]
  \emph{Der restliche Beweis mit dem Satz von Wilson}
\end{proof}

\newpage
\subsection{Aufgabe 3}
Sei $p_n$ die $n$-te Primzahl, d.\,h. $p_1 = 2$, $p_2 = 3$ usw. Zeigen Sie:
$p_n \leq 2^{2^{n - 1}}$ für alle $n \geq 1$.
\begin{proof}
\end{proof}

\newpage
\subsection{Aufgabe 4}

Sei $p$ eine Primzahl. Beweisen Sie: $p$ ist ein Teiler von $\binom{p}{v}$
für $1 \leq v \leq p - 1$.
\begin{proof}
  Per Definition gilt:
  \begin{equation*}
    \binom{p}{v} := \frac{p(p-1) \cdot \ldots \cdot (p - v + 1)}{v!}
  \end{equation*}
  Es gilt außerdem:
  \begin{equation*}
    \binom{n}{v} \in \SetN \quad \text{für alle} \enspace n,v \in \SetN
  \end{equation*}
  Die Primzerlegung des Nenners muss
  vollständig in der des Zählers vorhanden sein.
  Wegen $p > v$ ist $p$ jedoch niemals Teil dieser Zerlegung und
  kann im Zähler nicht gekürzt werden.
  Es folgt $p \mid \binom{p}{v}$.
\end{proof}

\newpage
\subsection{Aufgabe 5}
Seien $p \in \SetP{}, n \in \SetNx{}$ und $a,b \in \SetZ{}$. Zeigen Sie durch Induktion
nach $n$: $p$ ist ein Teiler von
$((a + b)^{p^n} - (a^{p^n} + b^{p^n}))$.
\begin{proof}
\end{proof}

\newpage
\subsection{Aufgabe 6}
Sei $n \geq 2$ eine natürliche Zahl. Zeigen Sie: $n^4 + 4^n$ ist keine Primzahl.
\begin{proof}
\end{proof}