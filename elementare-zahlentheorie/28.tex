\section{Seite 28}

\subsection{Aufgabe 1}
Seien $a,b,c$ Ziffern aus der Menge $\{0,1,2,\dotsc,9\}$ und $a \neq 0$.
Zeigen Sie: 13 teilt die natürliche Zahl $abcabc$ (Zifferndarstellung).
\begin{proof}
Es werden die Differenzen betrachtet, wenn sich $a, b, c$ um einen Wert verändern:
\begin{align*}
a &= 1 \Rightarrow a = 2: \triangle 100100 \\
b &= 0 \Rightarrow b = 1: \triangle 10010 \\
c &= 0 \Rightarrow c = 1: \triangle 1001
\end{align*}
Es ist zu sehen $13 \mid 1001$, denn es gilt: $1001 = 13 \cdot 77$.
Hieraus folgt: $13 \mid 10010, 13 \mid 100100$ und damit auch $13 \mid abcabc$. 
\end{proof}

\subsection{Aufgabe 2}
Sei $n$ eine natürliche Zahl, $n > 1$. Beweisen sie: Aus $n \mid (n - 1)! + 1$
folgt $n \in \MP$.
\begin{proof}
Ist $n$ eine zusammengesetze Zahl $n = ab$ mit $a,b > 1$, dann gilt
$a \mid (n - 1)!$ und dadurch $a \nmid (n - 1)! + 1$ weshalb ebenfalls
$n \nmid (n - 1)! + 1$. \footnote{Für $n > 4$ gilt außerdem: $n | (n - 1)!$.
Die gesamte Primzerlegung von $n$ ist in $(n - 1)!$ enthalten.} \\[4pt]
TODO: Der restliche Beweis mit dem Satz von Wilson
\end{proof}

\subsection{Aufgabe 3}
Sei $p_n$ die $n$-te Primzahl, d.\,h. $p_1 = 2$, $p_2 = 3$ usw. Zeigen Sie:
$p_n \leq 2^{2^{n - 1}}$ für alle $n \geq 1$.
\begin{proof}
TODO
\end{proof}

\subsection{Aufgabe 4}
Sei $p$ eine Primzahl. Beweisen Sie: $p$ ist ein Teiler von $\binom{p}{v}$
für $1 \leq v \leq p - 1$.
\begin{proof}
Obwohl die Zahlen $\binom{n}{v}$ als Brüche definiert sind, gilt stets:
\begin{equation*}
\binom{n}{v} \in \MN \quad \text{für alle } n,v \in \MN.
\end{equation*}
Dies ist auf die Identität
\begin{equation*}
\binom{n - 1}{v - 1} + \binom{n - 1}{v} = \binom{n}{v}
\end{equation*}
zurückzuführen, mit der sich jeder Binomialkoeffizient rekursiv als
die Summe natürlicher Zahlen berechnen lässt.
Per Definition gilt:
\begin{equation*}
\binom{p}{v} := \frac{p(p-1) \cdot \ldots \cdot (p - v + 1)}{v!}
\end{equation*}
Die Primzerlegung des Nenners muss vollständig in der des Zählers vorhanden
sein. Wegen $p > v$ ist $p$ jedoch niemals Teil dieser Zerlegung und
kann im Zähler nicht gekürzt werden.
Es folgt: $p \mid \binom{p}{v}$.
\end{proof}

\subsection{Aufgabe 5}
Seien $p \in \MP{}, n \in \MNx{}$ und $a,b \in \MZ{}$. Zeigen Sie durch Induktion
nach $n$: $p$ ist ein Teiler von
$((a + b)^{p^n} - (a^{p^n} + b^{p^n}))$.
\begin{proof}
  
\end{proof}

\subsection{Aufgabe 6}
Sei $n \geq 2$ eine natürliche Zahl. Zeigen Sie: $n^4 + 4^n$ ist keine Primzahl.
\begin{proof}
  
\end{proof}