\section[Über die Verteilung und Darstellung von Primzahlen (81)]
 {Über die Verteilung und Darstellung von\\Primzahlen (81)}

\subsection{Aufgabe 1}
Zeigen Sie mit Hilfe des Kriterium von \texttt{Lucas-Lehmer}
\parencite[78]{book:zahlentheorie},
dass die Zahlen $M_p = 2^p - 1$ für $p = 3$, $p = 5$ und
$p = 7$ Primzahlen sind.

\subsection{Aufgabe 2}
Zeigen Sie direkt: Es gibt unendliche Primzahlen der Form:
\begin{enumerate}[label=\alph*)]
  \item $6k + 5$, $k \in \swhole$
  \item $4k + 3$, $k \in \swhole$
\end{enumerate}

\begin{lemma}
  Sind $a_1,\dotsc,a_k$ Zahlen der Form $a_v = qb_v + 1$, $q,b_v \in \sintegers$,
  $1 \leq v \leq k$. Dann ist auch ihr Produkt $a_1a_2 \dotsm a_k$ von
  der Form $qb + 1$, $b \in \sintegers$.
\end{lemma}

\begin{proof}
  Der Beweis ergibt sich unmittelbar durch Induktion, da
  $(qm + 1)(qn + 1) = q(qmn + m + n) + 1$ für alle $m,n \in \sintegers$.
\end{proof}
\begin{proof}
  \begin{enumerate}[label=\alph*)]
    \item Angenommen es gäbe nur endlich viele Primzahlen der Form $6k + 5$,
          etwa $p_1,p_2,\dotsc,p_s$ mit $p_1 \defined 2$.
          Man betrachte die Zahl $a \defined (2 \cdot 3)(p_1p_2 \dotsm p_s) - 1
            \in \snatural$. Diese Zahl ist von der Form $6k + 5$.
          Sei $a = q_1^{m1}q_2^{m2} \dotsm q_t^{m_t}$ die Primzerlegung von $a$.
          Jede Primzahl $q_i$ ist von allen Primzahlen $2,3,p_1,\dotsc,p_s$
          verschieden, da $1$ nicht durch $q_i$ teilbar ist, $i = 1,2,\dots,t$.
          Nun muss aber mindestens eine Primzahl $q_i$ von der Form $6k + 5$ sein;
          denn alle anderen Möglichkeiten können wie folgt ausgeschlossen werden:
          \begin{enumerate}
            \item Die Primzahlen $q_i$ können nicht gerade oder
                  sonstig teilbar sein,
                  denn dann wären es keine Primzahlen.
                  \Ac{dh} die folgenden Formen fallen weg:
                  \begin{equation*}
                    6k + 0,\, 6k + 2 = 2(3k + 1),\, 6k + 3 = 3(2k + 1),\,
                    6k + 4 = 2(3k + 2)
                  \end{equation*}
            \item Zusätzlich können sie nicht alle in der Form $6k + 1$ sein; denn
                  so wäre nach dem eingangs bewiesenen auch die Zahl
                  $a = q_1^{m1}q_2^{m2} \dotsm q_t^{m_t}$ von
                  der Form $6k + 1$, was nicht möglich ist.
          \end{enumerate}
    \item Angenommen es gäbe nur endlich viele Primzahlen der Form $4k + 3$,
          etwa $p_1,p_2,\dotsc,p_s$ mit $p_1 \defined 2$.
          Man betrachte die Zahl $a \defined (2 \cdot 2)(p_1p_2 \dotsm p_s) - 1
            \in \snatural$. Diese Zahl ist von der Form $4k + 3$.
          Sei $a = q_1^{m1}q_2^{m2} \dotsm q_t^{m_t}$ die Primzerlegung von $a$.
          Jede Primzahl $q_i$ ist von allen Primzahlen $2,p_1,\dotsc,p_s$
          verschieden, da $1$ nicht durch $q_i$ teilbar ist, $i = 1,2,\dots,t$.
          Nun muss aber mindestens eine Primzahl $q_i$ von der Form $4k + 3$ sein;
          denn alle anderen Möglichkeiten können wie folgt ausgeschlossen werden:
          \begin{enumerate}
            \item Die Primzahlen $q_i$ können nicht gerade oder sonstig teilbar sein,
                  denn dann wären es keine Primzahlen.
                  \Ac{dh} die folgenden Formen fallen weg:
                  \begin{equation*}
                    4k + 0,\, 4k + 2 = 2(2k + 1)
                  \end{equation*}
            \item Zusätzlich können sie nicht alle in der Form $4k + 1$ sein; denn
                  so wäre nach dem eingangs bewiesenen auch die Zahl
                  $a = q_1^{m1}q_2^{m2} \dotsm q_t^{m_t}$ von
                  der Form $4k + 1$, was nicht möglich ist.
          \end{enumerate}
  \end{enumerate}
\end{proof}

\subsection{Aufgabe 3}
Zeigen Sie: Für alle natürlichen Zahlen $n \geq 2$ gilt:
$n < 2^{\primesequencecountfunc{n}} \cdot \sqrt{n}$.
