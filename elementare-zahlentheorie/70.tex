\section{(70) Größter gemeinsamer Teiler}

\subsection{Aufgabe 1}
Seinen $a,m,n \in \snatural$. Bestimmen Sie den größten gemeinsamen Teiler
von $a^m - 1$ und $a^n - 1$.
\begin{proof}
\end{proof}

\subsection{Aufgabe 2}
Seien $a,b \in \snatural$ teilerfremd und $c \in \swhole$ so,
dass gilt: $a \mid c$ und $b \mid c$. Zeigen Sie: $ab \mid c$.
\begin{proof}
  Das Kriterium für paarweise Teilerfremdheit \parencite[50]{book:zahlentheorie}
  enthält als einfache
  \begin{folgerung}
    \label{folgerung:70:1}
    Seien $a,b \in \sintegers$ zwei teilerfremde Zahlen, dann ist
    $\min{(\vielfachfunktion{a}, \vielfachfunktion{b})} = 0$
    für alle $p \in \sprime$.
  \end{folgerung}
  \begin{proof}
    Wäre die Vielfachheitsfunktions mit $p$ für beide $a,b > 0$.
    Dann ist genau dieses $p$ ein gemeinsamer Teiler von $a$ und $b$.
  \end{proof}
  \noindent
  Es gilt $\vielfachfunktion{a} \leq \vielfachfunktion{c}$,
  $\vielfachfunktion{b} \leq \vielfachfunktion{c}$ für alle $p \in \sprime$
  nach dem Teilbarkeitskriterium \parencite[50]{book:zahlentheorie}.
  Es ist $ab = \sum_{p} p^{\vielfachfunktion{a} + \vielfachfunktion{b}}$
  mit $p \in \sprime$ die Primzerlegung von $ab$.
  Da $a$ und $b$ teilerfremd sind, gilt nach
  \autoref{folgerung:70:1} $\vielfachfunktion{a} + \vielfachfunktion{b}
    \leq \vielfachfunktion{c}$. Es folgt $ab \mid c$.
\end{proof}

\subsection{Aufgabe 3}
Seien $a,b \in \snatural$. Zeigen Sie: $\ggt{a + b}{a - b} \geq \ggt{a}{b}$.
\begin{proof}
\end{proof}

\subsection{Aufgabe 4}
\label{70:4}
Seien $a,b,m \in \sintegers$. Zeigen Sie die Äquivalenz folgender Aussagen:
\begin{enumerate}[label=\roman*)]
  \item Es gibt eine ganze Zahl $x$ mit $m \mid (ax - b)$
  \item $\ggt{a}{m} \mid b$
\end{enumerate}
\begin{proof}
  \implication{1}{2}: Sei $t = \ggt{a}{m}$. Es gilt $t \mid a$, $t \mid m$ und
  aus letzterem $t \mid ax - b$. Weiter gilt nach den Rechenregeln zur Teilbarkeit
  $t \mid b$ und dies erledigt die Beweisrichtung.\\
  \implication{2}{1}: $\ggt{a}{m}$ liefert die Gleichung $t = ra + sm$
  mit $r,s \in \sintegers$. Aus $\ggt{a}{m} \mid b$ folgt mit $v \in \sintegers$:
  \begin{equation*}
    \begin{aligned}
      b      & = tv = rva + svm               \\
      svm    & = b - rva \qquad x \defined rv \\
      (-sv)m & = ax - b
    \end{aligned}
  \end{equation*}
  Daher $m \mid (ax - b)$.
\end{proof}

\subsection{Aufgabe 5}
Seien $m,n \in \sintegers$ teilerfremd, $k \defined mn$ sowie
$a,b \in \sintegers$ beliebig. Zeigen Sie (unter Verwendung von \autoref{70:4}):
\begin{enumerate}[label=\alph*)]
  \item Es gibt eine ganze Zahl $u$ mit $m \mid (u - a)$ und $n \mid (u - b)$
  \item Für eine ganze Zahl $x$ sind äquivalent:
        \begin{enumerate}[label=\roman*)]
          \item $m \mid (x - a)$ und $n \mid (x - b)$
          \item $k \mid (x - u)$
        \end{enumerate}
\end{enumerate}
\begin{proof}
  \begin{enumerate}[label=\alph*)]
    \item Es ist $\textcolor{Red}{p}m = u - a$ und
          $\textcolor{Green}{q}n = u - b$ mit $p,q \in \sintegers$.
          D.\,h. $u$ ist die Lösung der Gleichung
          $\textcolor{Red}{p}m - \textcolor{Green}{q}n = b - a$.
          Nach Voraussetzung existiert $rm + sn = 1$ mit $r,s \in \sintegers$.
          Bemerke die Terme $rm$ und $sn$ haben zwangsweise
          unterschiedliche Vorzeichen. Nach Multiplikation mit
          $b - a$ entsteht daher
          \begin{equation}
            \label{70:1}
            \textcolor{Red}{(br - ar)}m + \textcolor{Green}{(bs - as)}n = b - a.
          \end{equation}
          Also, es existiert ein $u$ mit
          \begin{equation*}
            u = \textcolor{Red}{(br - ar)}m + a = \textcolor{Green}{(bs - as)}n + b.
          \end{equation*}
    \item \autoref{70:1} gibt eine Lösung für $u$.
          Es ist leicht hierdurch alle anderen Lösungen anzugeben.
          Mit dem Wissen des Vorzeichenverhaltens von \eqref{70:1},
          rechne man mit $v \in \sintegers$ wie folgt:
          \begin{equation*}
            \textcolor{Red}{(br - ar + vn)}m + \textcolor{Green}{(bs - as + vm)}n
            = b - a + (vmn - vmn)
          \end{equation*}
          Sind also $x = \textcolor{Red}{(br - ar + v_1n)}m + a$ und
          $u = \textcolor{Red}{(br - ar + v_2n)}m + a$ mit $v_1 \neq v_2 \in \sintegers$
          zwei Lösungen der Teilbarkeit,
          dann ist $x - u = v_1mn - v_2mn = (v_1 - v_2)mn$.
          Mit $k \defined mn$ gilt also $k \mid x - u$.
  \end{enumerate}
\end{proof}

\subsection{Aufgabe 6}
\begin{enumerate}[label=\alph*)]
  \item Seien $\mathfrak{a},\mathfrak{b}$ zwei Ideale in $\sintegers$. Zeigen Sie:
        $\mathfrak{a} \cap \mathfrak{b}$ ist wieder ein Ideal in $\sintegers$.
  \item Zeigen Sie: Für ganze Zahlen $a,b,v$ sind folgende Aussagen äquivalent:
        \begin{enumerate}[label=\roman*)]
          \item $v \geq 0$ und
                $\idealintergers{v} = \idealintergers{a} \cap \idealintergers{b}$
          \item $v = \kgv{a}{b}$
        \end{enumerate}
\end{enumerate}

\begin{lemma}
  \label{lemma:70.6.1}
  Es seien $a,b \in \sintegers$ zwei Zahlen derart, dass für das von ihnen
  erzeugte Hauptideal $\idealintergers{a}, \idealintergers{b}$ gilt:
  $\idealintergers{a} \subseteq \idealintergers{b}$.
  Dann ist notwendigerweise $b \mid a$.
\end{lemma}
\begin{proof}
  Es wird \obda{} angenommen $a,b \geq 0$.
  Der Fall für $b = a$ und $a = 0$ ist klar. Ist $b = 0$, so muss $a$
  als einzige Teilmenge ebenfalls $0$ sein. Es kann niemals gelten $b > a$, denn
  dann wäre $a \notin \idealintergers{b}$. Es muss also sein $b < a$.
  Es ist $a \in \idealintergers{b}$ und
  es gilt somit die Gleichung $bx = a$ mit $x \in \sintegers$.
  Es folgt $b \mid a$.
\end{proof}
\begin{proof}
  \begin{enumerate}[label=\alph*)]
    \item Wir zeigen, dass $\mathfrak{a} \cap \mathfrak{b}$ die Bedingungen
          der Definition eines Ideals in $\sintegers$ erfüllt
          \parencite[60]{book:zahlentheorie}.\\
          \ad{1} Angenommen $a,b \in \mathfrak{a} \cap \mathfrak{b}$.
          Es wird gezeigt, dass auch $a - b \in \mathfrak{a} \cap \mathfrak{b}$.
          Per Annahme wissen wir $a,b \in \mathfrak{a}$ und $a,b \in \mathfrak{b}$.
          Nach Idealdefinition ist somit ebenfalls
          $a - b \in \mathfrak{a}$ und $a - b \in \mathfrak{b}$.
          Es folgt $a - b \in \mathfrak{a} \cap \mathfrak{b}$.\\
          \ad{2} Angenommen $a \in \mathfrak{a} \cap \mathfrak{b}$.
          Es wird gezeigt, dass auch $xa \in \mathfrak{a} \cap \mathfrak{b}$
          mit $x \in \sintegers$. Per Annahme wissen wir
          $a \in \mathfrak{a}$ und $a \in \mathfrak{b}$.
          Nach Idealdefinition ist somit ebenfalls
          $xa \in \mathfrak{a}$ und $xa \in \mathfrak{b}$.
          Es folgt $xa \in \mathfrak{a} \cap \mathfrak{b}$.
    \item Aus $\idealintergers{v} = \idealintergers{a} \cap \idealintergers{b}$
          folgt $\idealintergers{v} \subseteq \idealintergers{a}$,
          $\idealintergers{v} \subseteq \idealintergers{b}$ und nach \autoref{lemma:70.6.1}
          also $a | v$ und $b | v$. Die Zahl $v$ ist somit
          ein gemeinsames Vielfaches von $a$ und $b$.
          Angenommen $c$ ist ein weiteres gemeinsames Vielfaches von $a$ und $b$,
          dann gilt $a | c$ und $b | c$.
          Wieder nach \autoref{lemma:70.6.1} ist also
          $\idealintergers{c} \subseteq \idealintergers{a}$ und
          $\idealintergers{c} \subseteq \idealintergers{b}$.
          Es folgt die logische Schlussfolgerung
          \begin{equation*}
            (\idealintergers{c} \subseteq \idealintergers{a}) \wedge
            (\idealintergers{c} \subseteq \idealintergers{a}) \wedge
            (\idealintergers{a} \cap \idealintergers{b} = \idealintergers{v})
            \Rightarrow \idealintergers{c} \subseteq \idealintergers{v}.
          \end{equation*}
          Es gilt also $v \mid c$, sowie $a \mid v$ und $b \mid v$.
          Die Zahl $v$ erfüllt somit alle Eingenschaften des
          kleinsten gemeinsamen Vielfaches von $a$ und $b$.
  \end{enumerate}
\end{proof}

\subsection{Aufgabe 7}
Seien $a,b,c \in \snatural$. Zeigen Sie: Es gilt $a^2 + b^2 = c^2$ genau dann,
wenn es $s,u,v \in \snatural$ mit $u > v$ gibt, sodass entweder
$a = 2suv$, $b = s(u^2 - v^2)$, $c = s(u^2 + v^2)$ oder
$a = s(u^2 + r^2)$, $b = 2suv$, $c = s(u^2 + v^2)$.
\begin{proof}
\end{proof}
