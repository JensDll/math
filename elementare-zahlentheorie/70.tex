\section{Seite 70}

\subsection{Aufgabe 1}
Seinen $a,m,n \in \NaturalNumbers{}$. Bestimmen Sie den größten gemeinsamen Teiler
von $a^m - 1$ und $a^n - 1$.
\begin{proof}
\end{proof}

\subsection{Aufgabe 2}
Seien $a,b \in \NaturalNumbers{}$ teilerfremd und $c \in \WholeNumbers{}$ so,
dass gilt: $a \mid c$ und $b \mid c$. Zeigen Sie: $(ab) \mid c$.
\begin{proof}
  Das Kriterium für paarweise Teilerfremdheit \parencite[50]{book:zahlentheorie}
  enthält als triviale
  \begin{folgerung}
    \label{folgerung:70:1}
    Seien $a,b \in \Integers{}$ zwei teilerfremde Zahlen, dann ist
    $\min{(\vielfachfunktion{a}, \vielfachfunktion{b})} = 0$
    für alle $p \in \PrimeNumbers{}$.
  \end{folgerung}
  \noindent
  Es gilt $\vielfachfunktion{a} \leq \vielfachfunktion{c}$,
  $\vielfachfunktion{b} \leq \vielfachfunktion{c}$ für alle $p \in \PrimeNumbers{}$
  nach dem Teilbarkeitskriterium \parencite[50]{book:zahlentheorie}.
  Es ist $ab = \sum_{p} p^{\vielfachfunktion{a} + \vielfachfunktion{b}}$
  die Primzerlegung von $ab$. Da $a$ und $b$ teilerfremd sind, gilt nach
  \autoref{folgerung:70:1} $\vielfachfunktion{a} + \vielfachfunktion{b}
    \leq \vielfachfunktion{c}$. Es folgt $(ab) \mid c$.
\end{proof}

\subsection{Aufgabe 3}
Seien $a,b \in \NaturalNumbers{}$. Zeigen Sie: $\ggt{a + b}{a - b} \geq \ggt{a}{b}$.
\begin{proof}
\end{proof}

\subsection{Aufgabe 4}
\label{70:4}
Seien $a,b,m \in \Integers{}$. Zeigen Sie die Äquivalenz folgender Aussagen:
\begin{enumerate}[label=\roman*)]
  \item Es gibt eine ganze Zahl $x$ mit $m \mid (ax - b)$
  \item $\ggt{a}{m} \mid b$
\end{enumerate}
\begin{proof}
  \implication{1}{2}: Sei $t = \ggt{a}{m}$. Es gilt $t \mid a$, $t \mid m$ und
  daher $t \mid ax - b$. Weiter gilt $t \mid -b$ und dies erledigt die Beweisrichtung.\\
  \implication{2}{1}: $\ggt{a}{m}$ liefert die Gleichung $t = ra + sm$
  mit $r,s \in \Integers{}$. Aus $\ggt{a}{m} \mid b$ folgt mit $v \in \Integers{}$:
  \begin{equation*}
    \begin{aligned}
      b      & = tv = rva + svm                                              \\
      svm    & = b - rva \qquad\quad \text{definiere} \enspace x \defined rv \\
      (-sv)m & = ax - b
    \end{aligned}
  \end{equation*}
  Daher $m \mid (ax - b)$.
\end{proof}

\subsection{Aufgabe 5}
Seien $m,n \in \Integers{}$ teilerfremd, $k \defined{} mn$ sowie
$a,b \in \Integers{}$ beliebig. Zeigen Sie (unter Verwendung von \autoref{70:4}):
\begin{enumerate}[label=\alph*)]
  \item Es gibt eine ganze Zahl $u$ mit $m \mid (u - a)$ und $n \mid (u - b)$
  \item Für eine ganze Zahl $x$ sind äquivalent:
        \begin{enumerate}[label=\roman*)]
          \item $m \mid (x - a)$ und $n \mid (x - b)$
          \item $k \mid (x - u)$
        \end{enumerate}
\end{enumerate}
\begin{proof}
  \begin{enumerate}[label=\alph*)]
    \item Es ist $\textcolor{Red}{p}m = u - a$ und
          $\textcolor{Green}{q}n = u - b$ mit $p,q \in \Integers{}$.
          D.\,h. $u$ ist die Lösung der Gleichung
          $\textcolor{Red}{p}m - \textcolor{Green}{q}n = b - a$.
          Nach Voraussetzung $m \coprime n$ existiert $rm + sn = 1$
          mit $r,s \in \Integers{}$.
          Bemerke die Terme $rm$ und $sn$ haben zwangsweise unterschiedliche Vorzeichen.
          Daher:
          \begin{equation}
            \label{70:1}
            \textcolor{Red}{(br - ar)}m + \textcolor{Green}{(bs - as)}n = b - a
          \end{equation}
          Also es existiert ein $u$ mit der Lösung:
          \begin{equation*}
            u = \textcolor{Red}{(br - ar)}m + a = \textcolor{Green}{(bs - as)}n + b
          \end{equation*}
    \item \iimplication{1}{2} \autoref{70:1} gibt eine Lösung für $u$.
          Es ist leicht hierdurch alle anderen Lösungen anzugeben.
          Mit dem Wissen des Vorzeichenverhaltens von \eqref{70:1},
          rechne man mit $v \in \Integers{}$ wie folgt:
          \begin{equation*}
            \textcolor{Red}{(br - ar + vn)}m + \textcolor{Green}{(bs - as + vm)}n
            = b - a + (vmn - vmn)
          \end{equation*}
          Sind also $x = \textcolor{Red}{(br - ar + v_1n)}m + a$ und
          $u = \textcolor{Red}{(br - ar + v_2n)}m + a$ mit $v_1 \neq v_2 \in \Integers{}$
          zwei Lösungen der Teilbarkeit,
          dann ist $x - u = v_1mn - v_2mn = (v_1 - v_2)mn$.
          Mit $k \defined mn$ gilt also $k \mid x - u$.
  \end{enumerate}
\end{proof}

\subsection{Aufgabe 6}
\begin{enumerate}[label=\alph*)]
  \item Seien $a,b$ zwei Ideale in $\Integers{}$. Zeigen Sie:
        $a \cap b$ ist wieder ein Ideal in $\Integers{}$.
  \item Zeigen Sie: Für ganze Zahlen $a,b,v$ sind folgende Aussagen äquivalent:
        \begin{enumerate}[label=\roman*)]
          \item $v \geq 0$ und
                $\IdealIntegers{v} = \IdealIntegers{a} \cap \IdealIntegers{b}$
          \item $v = \kgv{a}{b}$
        \end{enumerate}
\end{enumerate}
\begin{proof}
  \begin{enumerate}[label=\alph*)]
    \item Jedes Ideal in $\Integers{}$ enthält die Null.
          Denn $0 \in (a_1,a_2,\dotsc,a_n)$, da $a_1 = a_2 = \dotsb
            = a_n = 0a_1 + 0 a_2 + \dotsb + 0a_n$.
          Jede Schnittmenge enthält somit ebenfalls die Null
          und ist also mindestens ein Nullideal.
    \item \iimplication{1}{2} Die Schnittmenge $\Integers{}a \cap \Integers{}b$ enthält
          alle Lösungen der Gleichung $pa = qb$ mit $p,q \in \Integers{}$:
          \begin{equation*}
            \Integers{}a \cap \Integers{}b = \{z \in \Integers{}: z = pa \enspace
            \text{und} \enspace pa = qb \enspace
            \text{für alle} \enspace p,q \in \Integers{}\}
          \end{equation*}
          Die kleinste positive Lösung $pa = qb > 0$ bei gegebenen $a,b$ ist offensichtlich
          $\kgv{a}{b}$. Alle weiteren Lösungen sind Vielfache dieser Lösung
          und es entsteht die Lösungsmenge:
          \begin{equation*}
            \{z \in \Integers{}: z = n \cdot \kgv{a}{b}
            \enspace \text{mit} \enspace n \in \Integers{} \}
          \end{equation*}
          Dies entspricht der Definition von $\Integers{}\kgv{a}{b}$.
          Daher sind äquivalent
          $v = \kgv{a}{b}$ und $\IdealIntegers{v} = \IdealIntegers{a} \cap \IdealIntegers{b}$.
  \end{enumerate}
\end{proof}

\subsection{Aufgabe 7}
Seien $a,b,c \in \NaturalNumbers{}$. Zeigen Sie: Es gilt $a^2 + b^2 = c^2$ genau dann,
wenn es $s,u,v \in \NaturalNumbers{}$ mit $u > v$ gibt, sodass entweder
$a = 2suv$, $b = s(u^2 - v^2)$, $c = s(u^2 + v^2)$ oder
$a = s(u^2 + r^2)$, $b = 2suv$, $c = s(u^2 + v^2)$.
\begin{proof}
\end{proof}
