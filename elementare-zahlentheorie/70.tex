\section{Seite 70}

\subsection{Aufgabe 1}
Seinen $a,m,n \in \NaturalNumbers{}$. Bestimmen Sie den größten gemeinsamen Teiler
von $a^m - 1$ und $a^n - 1$.
\begin{proof}
\end{proof}

\subsection{Aufgabe 2}
Seien $a,b \in \NaturalNumbers{}$ teilerfremd und $c \in \WholeNumbers{}$ so,
dass gilt: $a \mid c$ und $b \mid c$. Zeigen Sie: $(ab) \mid c$.
\begin{proof}
  Das Kriterium für paarweise Teilerfremdheit \parencite[50]{book:zahlentheorie}
  enthält als triviale
  \begin{folgerung}
    \label{folgerung:70:1}
    Seien $a,b \in \Integers{}$ zwei teilerfremde Zahlen, dann ist
    $\min{(\vielfachfunktion{a}, \vielfachfunktion{b})} = 0$
    für alle $p \in \PrimeNumbers{}$.
  \end{folgerung}
  \noindent
  Es gilt $\vielfachfunktion{a} \leq \vielfachfunktion{c}$,
  $\vielfachfunktion{b} \leq \vielfachfunktion{c}$ für alle $p \in \PrimeNumbers{}$
  nach dem Teilbarkeitskriterium \parencite[50]{book:zahlentheorie}.
  Es ist $ab = \sum_{p} p^{\vielfachfunktion{a} + \vielfachfunktion{b}}$
  die Primzerlegung von $ab$. Da $a$ und $b$ teilerfremd sind, gilt nach
  \autoref{folgerung:70:1} $\vielfachfunktion{a} + \vielfachfunktion{b}
    \leq \vielfachfunktion{c}$. Es folgt $(ab) \mid c$.
\end{proof}

\subsection{Aufgabe 3}
Seien $a,b \in \NaturalNumbers{}$. Zeigen Sie: $\ggt{a + b}{a - b} \geq \ggt{a}{b}$.
\begin{proof}
\end{proof}

\subsection{Aufgabe 4}
\label{70:4}
Seien $a,b,m \in \Integers{}$. Zeigen Sie die Äquivalenz folgender Aussagen:
\begin{enumerate}[label=\roman*)]
  \item Es gibt eine ganze Zahl $x$ mit $m \mid (ax -b)$
  \item $\ggt{a}{m} \mid b$
\end{enumerate}

\subsection{Aufgabe 5}
Seien $m,n \in \Integers{}$ teilerfremd, $k \defined{} mn$ sowie
$a,b \in \Integers{}$ beliebig. Zeigen Sie (unter Verwendung von \autoref{70:4}):
\begin{enumerate}[label=\alph*)]
  \item Es gibt eine ganze Zahl $u$ mit $m \mid (u - a)$ und $n \mid (u - b)$
  \item Für eine ganze Zahl $x$ sind äquivalent:
        \begin{enumerate}[label=\roman*)]
          \item $m \mid (x - a)$ und $n \mid (x - b)$
          \item $k \mid (x - a)$
        \end{enumerate}
\end{enumerate}
\begin{proof}
\end{proof}

\subsection{Aufgabe 6}
\begin{enumerate}[label=\alph*)]
  \item Seien $a,b$ zwei Ideale in $\Integers{}$. Zeigen Sie:
        $a \cap b$ ist wieder ein Ideal in $\Integers{}$.
  \item Zeigen Sie: Für ganze Zahlen $a,b,v$ sind folgende Aussagen äquivalent:
        \begin{enumerate}[label=\roman*)]
          \item $v \geq 0$ und
                $\IdealIntegers{v} = \IdealIntegers{a} \cap \IdealIntegers{b}$
          \item $v = \kgv{a}{b}$
        \end{enumerate}
\end{enumerate}
\begin{proof}
\end{proof}

\subsection{Aufgabe 7}
Seien $a,b,c \in \NaturalNumbers{}$. Zeigen Sie: Es gilt $a^2 + b^2 = c^2$ genau dann,
wenn es $s,u,v \in \NaturalNumbers{}$ mit $u > v$ gibt, sodass entweder
$a = 2suv$, $b = s(u^2 - v^2)$, $c = s(u^2 + v^2)$ oder
$a = s(u^2 + r^2)$, $b = 2suv$, $c = s(u^2 + v^2)$.
\begin{proof}
\end{proof}
