\section{(53) Anwendung des Hauptsatzes}

\subsection{Aufgabe 1}
Sei $p$ eine Primzahl, $a,b$ seien von Null verschiedene rationale Zahlen,
$a + b \neq 0$. Zeigen Sie: $\vielfachfunktion{a + b} \geq \min{(\vielfachfunktion{a}, \vielfachfunktion{b})}$
\begin{proof}
  Sei $m = \min{(\vielfachfunktion{a}, \vielfachfunktion{b})}$. Es gilt $p^m \mid a$, $p^m \mid b$
  und damit auch $p^m \mid a + b$. Wir schreiben $a + b = p^{m}v$
  mit $v \in \sintegers$ und zeigen durch umformen:
  \begin{align*}
    \vielfachfunktion{a + b} & = \vielfachfunktion{p^{m}v}                                                 \\
                             & = \vielfachfunktion{p^m} + \vielfachfunktion{v}                             \\
                             & = m + \vielfachfunktion{v}                                                  \\
                             & = \min{(\vielfachfunktion{a}, \vielfachfunktion{b})} + \vielfachfunktion{v}
  \end{align*}
  Es ist zu sehen $\vielfachfunktion{a + b} \geq \min{(\vielfachfunktion{a}, \vielfachfunktion{b})}$.
\end{proof}

\subsection{Aufgabe 2}
Für $x$ reell bezeichne $\floor{x}$ die größte ganze Zahl $m$ mit $m \leq x$.
Zeigen Sie, dass für $p$ eine Primzahl und $n \in \swhole$ beliebig gilt:
\begin{equation*}
  \vielfachfunktion{n!} = \sum_{i=0}^{\infty} \floor*{\frac{n}{p^i}}
\end{equation*}
\begin{proof}
\end{proof}

\subsection{Aufgabe 3}
Seien $n \in \snatural$, $a_1,\dotsc,a_n \in \sintegers$.
Die reelle Zahl $x$ erfülle $x^n + a_1x^{n-1} + \dotsb + a_{n - 1}x + a_n = 0$.
Zeigen Sie: $x$ ist entweder irrational oder ganz.
\begin{proof}
\end{proof}

\subsection{Aufgabe 4}
Seien $q_1,\dotsc,q_s$ Primzahlen, $b \defined q_1 \cdot q_2 \dotsm q_s \in \swhole$
sowie $m_1,\dotsc,m_k \in \snatural$ derart, dass gilt:
$\frac{1}{b} = \frac{1}{m_1} + \frac{1}{m_2} + \dotsb + \frac{1}{m_k}$.
Zeigen Sie: Jede Zahl $q_i, 1 \leq i \leq s$, teilt wenigstens eine der Zahlen $m_1,\dotsc,m_k$.
\begin{proof}
  Durch Hauptnennerdarstellung entsteht mit $v \defined \frac{m_1 \dotsm m_r}{m_1} +
    \frac{m_1 \dotsm m_r}{m_2} + \dotsb + \frac{m_1 \dotsm m_r}{m_r}$
  die folgende Gleichung:
  \begin{equation*}
    \begin{aligned}
           &  & \frac{1}{b} & = \frac{v}{m_1m_2 \dotsm m_k} \\
      \iff &  & bv          & = m_1m_2 \dotsm m_k
    \end{aligned}
  \end{equation*}
  Es ist zu sehen $b \mid m_1m_2 \dotsm m_k$ und damit die zu zeigende Aussage.
\end{proof}

\subsection{Aufgabe 5}
Berechnen Sie die Fibonaccidarstellung des Bruches $\frac{21}{23}$.
\begin{proof}
  $n_1 = \min{\{ w \in \swhole : w \geq \frac{23}{21} \}} = 2$.
  Der größte in $\frac{21}{23}$ enthaltene Stammbruch ist $\frac{1}{2}$:
  \begin{equation*}
    \frac{21}{23} = \frac{1}{2} + \frac{a_1}{b_1} \quad \text{mit}
    \quad \frac{a_1}{b_1} = \frac{21}{23} - \frac{1}{2} =
    \frac{2 \cdot 21 - 23}{2 \cdot 23} = \frac{19}{46}.
  \end{equation*}
  $n_2 = \min{\{ w \in \swhole : w \geq \frac{46}{19} \}} = 3$.
  Der größte in $\frac{19}{46}$ enthaltene Stammbruch ist $\frac{1}{3}$:
  \begin{equation*}
    \frac{19}{46} = \frac{1}{3} + \frac{a_2}{b_2} \quad \text{mit}
    \quad \frac{a_2}{b_2} =
    \frac{3 \cdot 19 - 46}{3 \cdot 46} = \frac{11}{138}.
  \end{equation*}
  $n_3 = \min{\{ w \in \swhole : w \geq \frac{138}{11} \}} = 13$.
  Der größte in $\frac{11}{138}$ enthaltene Stammbruch ist $\frac{1}{13}$:
  \begin{equation*}
    \frac{11}{148} = \frac{1}{13} + \frac{a_3}{b_3} \quad \text{mit}
    \quad \frac{a_3}{b_3} =
    \frac{13 \cdot 11 - 138}{13 \cdot 138} = \frac{5}{1794}.
  \end{equation*}
  $n_4 = \min{\{ w \in \swhole : w \geq \frac{1794}{5} \}} = 359$.
  Der größte in $\frac{5}{1794}$ enthaltene Stammbruch ist $\frac{1}{359}$:
  \begin{equation*}
    \frac{5}{1794} = \frac{1}{359} + \frac{a_4}{b_4} \quad \text{mit}
    \quad \frac{a_4}{b_4} =
    \frac{359 \cdot 5 - 1794}{359 \cdot 1794} = \frac{1}{644046}.
  \end{equation*}
  Die Fibonaccidarstellung des Bruches $\frac{21}{23}$ lautet:
  \begin{equation*}
    \frac{21}{23} = \frac{1}{2} + \frac{1}{3} +
    \frac{1}{13} + \frac{1}{359} + \frac{1}{644046}
  \end{equation*}
\end{proof}

\subsection{Aufgabe 6}
Zeigen Sie: Es gibt keine ägyptische Bruchdarstellung
$\frac{21}{23} = \frac{1}{n_1} + \frac{1}{n_2} + \dotsb + \frac{1}{n_k}$,
$1 < n_1 < n_2 < \dotsb < n_k$, mit höchstens 3 Stammbrüchen
(d.\,h. notwendig $k \geq 4$).
\begin{proof}
\end{proof}

\subsection{Aufgabe 7}
Beweisen Sie die angegebene Eindeutigkeitsaussage für die Fibonaccidarstellung
\parencite[53]{book:zahlentheorie}.
\begin{proof}
\end{proof}
