\section{Seite 53}

\subsection{Aufgabe 1}
Sei $p$ eine Primzahl, $a,b$ seien von Null verschiedene rationale Zahlen,
$a + b \neq 0$. Zeigen Sie: $\vielfachfunktion{a + b} \geq \min{(\vielfachfunktion{a}, \vielfachfunktion{b})}$
\begin{proof}
  Sei $m = \min{(\vielfachfunktion{a}, \vielfachfunktion{b})}$. Es gilt $p^m \mid a$, $p^m \mid b$
  und damit auch $p^m \mid a + b$. Wir schreiben $a + b = p^m \cdot v$
  und zeigen durch umformen:
  \begin{align*}
    \vielfachfunktion{a + b} & = \vielfachfunktion{p^m \cdot v}                                            \\
                             & = \vielfachfunktion{p^m} + \vielfachfunktion{v}                             \\
                             & = m + \vielfachfunktion{v}                                                  \\
                             & = \min{(\vielfachfunktion{a}, \vielfachfunktion{b})} + \vielfachfunktion{v}
  \end{align*}
  Es ist zu sehen $\vielfachfunktion{a + b} \geq \min{(\vielfachfunktion{a}, \vielfachfunktion{b})}$.
\end{proof}

\subsection{Aufgabe 2}
Für $x$ reell bezeichne $\floor{x}$ die größte ganze Zahl $m$ mit $m \leq x$.
Zeigen Sie, dass für $p$ eine Primzahl und $n \in \WholeNumbers{}$ beliebig gilt:
\begin{equation*}
  \vielfachfunktion{n!} = \sum_{i=0}^{\infty} \floor*{\frac{n}{p^i}}
\end{equation*}
\begin{proof}
\end{proof}

\subsection{Aufgabe 3}
Seien $n \in \NaturalNumbers{}$, $a_1,\dotsc,a_n \in \Integers{}$.
Die reelle Zahl $x$ erfülle $x^n + a_1x^{n-1} + \dotsb + a_{n - 1}x + a_n = 0$.
Zeigen Sie: $x$ ist entweder irrational oder ganz.
\begin{proof}
\end{proof}

\subsection{Aufgabe 4}
Seien $q_1,\dotsc,q_s$ Primzahlen, $b \defined{} q_1 \cdot q_2 \dotsm q_s \in \WholeNumbers{}$
sowie $m_1,\dotsc,m_k \in \NaturalNumbers{}$ derart, dass gilt:
$\frac{1}{b} = \frac{1}{m_1} + \frac{1}{m_2} + \dotsb + \frac{1}{m_k}$.
Zeigen Sie: Jede Zahl $q_i, 1 \leq i \leq s$, teilt wenigstens eine der Zahlen $m_1,\dotsc,m_k$.
\begin{proof}
\end{proof}

\subsection{Aufgabe 5}
Berechnen Sie die Fibonaccidarstellung des Bruches $\frac{21}{23}$.
\begin{proof}
\end{proof}

\subsection{Aufgabe 6}
Zeigen Sie: Es gibt keine ägyptische Bruchdarstellung
$\frac{21}{23} = \frac{1}{n_1} + \frac{1}{n_2} + \dotsb + \frac{1}{n_k}$,
$1 < n_1 < n_2 < \dotsb < n_k$, mit höchstens 3 Stammbrüchen
(d.\,h. notwendig $k \geq 4$).
\begin{proof}
\end{proof}

\subsection{Aufgabe 7}
Beweisen Sie die angegebene Eindeutigkeitsaussage für die Fibonaccidarstellung
\parencite[53]{book:zahlentheorie}.
\begin{proof}
\end{proof}
